\begin{abstract}
We present \RM{}, a full-likelihood dynamical modelling machinery that aims to recover the Milky Way's (MW) gravitational potential from stellar sub-populations in the Galactic disk. \RM{} models the observed positions and velocities of stars with an equilibrium, three-integral distribution function (DF) in an axisymmetric potential and accounts also for survey selection effects. In preparation for the application to large data sets of modern surveys like Gaia, we create and analyze a large suite of mock data sets. Based on this we develop qualitative ``rules of thumb'' for which characteristics and limitations of data, model and machinery affect constraints on the potential and DF most. Overall we find that the potential can be reliably recovered if the model assumptions are fulfilled or even slightly wrong. \RM{} gives constraints of high precision (i) for large sample sizes, (ii) for survey volumes of large radial and vertical coverage, and (iii) as long as measurement uncertainties are perfectly known (even for proper motion uncertainties up to $\delta \mu \sim 5~\text{mas yr}^{-1}$). Unbiased potential estimates are ensured, (i) for small to moderate misjudgements of the spatial selection function, (ii) if distances are known to within $10\%$ (at least for distances smaller $3~\text{kpc}$ and $\delta \mu \lesssim 2~\text{mas yr}^{-1}$), and (iii) if proper motion uncertainties are known within $10\%$ (at least for $\delta \mu \lesssim 3~\text{mas yr}^{-1}$). Minor differences between the true and assumed DF are acceptable. When defining sub-populations by binning stars according to their chemical abundances, finite bin sizes and abundance errors should not affect the modelling as long as the DF parameters of neighbouring bins do not vary more than $20\%$. While hotter populations are less affected by pollution and misjudgements of $\delta \mu$, cooler populations recover the Galactic rotation curve more reliably. If the MW's true gravitational potential is not included in the assumed family of parametrized model potentials, we can---at least in the axisymmetric case---still find a potential that is a reliable fit within the limitations of the model. Challenges of the future are the rapidly increasing computational costs for high precision likelihood evaluations required for large sample sizes.
\end{abstract}





