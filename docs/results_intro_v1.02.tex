We are now in a position to explore the questions about the ultimate limitations of action based modelling, posed in the introduction: 
\begin{itemize}
\item Can we still retrieve unbiased  model parameter estimates $p_M$ in the limit of large sample sizes? 
\item What role does the survey volume and geometry play, at given sample size? 
\item What if our knowledge of the sample selection function is imperfect, and potentially biased? 
\item How do the parameter estimates deteriorate if the individual errors on the phase-space coordinates become significant? 
\end{itemize}
But we also consider the more fundamental limitations:
\begin{itemize}
\item What if the observed stars are not extacly drawn from the family of model distribution functions? 
\item What happens to the estimate of the potential and the DF, if the actual potential is not contained in the family of model potentials?
\end{itemize}
We do not explore the breakdown of the assumption that the system is axisymmetric and in steady state.
[{\bf hat shouldl also be at the end of the introduction..}
[say: except for the case of ``errors'' we assume that thne phase-space errors are negligible..]