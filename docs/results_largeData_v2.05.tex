\subsection{Model parameter estimates in the limit of large data sets} \label{sec:largedata}

The individual \MAP{}s in BR13 contained typically between 100 and 800 objects, so that each \MAP{} implied a quite broad \pdf{} for the model parameters $\pmodel{}$. Here we explore what happens in the limit of much larger samples, say $N_{*} = 20,000$ objects. As outlined in Section \ref{sec:likelihood} the immediate consequence of larger samples is given by the likelihood normalization requirement, $\log(1+\delta M_\text{tot})\le 1/N_{*}$ (see Equation \ref{eq:loglikelihood_relerr}), which is the modelling aspect that drives the computing time. This issues aside, we would, however, expect that in the limit of large data sets with vanishing measurement uncertainties the \pdf{}s of the \pmodel{} become Gaussian, with a \pdf{} width that scales as $1/\sqrt{N_{*}}$. Further, we must verify that any bias in the \pdf{} expectation value is considerably less than the error, even for quite large samples.

Using sets of mock data, created according to Section \ref{sec:mockdata} and a fiducial model for \pmodel{} (see Table \ref{tbl:tests}, Tests \ref{test:sqrtNiso}, \ref{test:isoSph_CLT}, and \ref{test:isoSphFlex}), we verified that \RM{} satisfies all these conditions and expectations: Figure \ref{fig:isoSphFlex_triangleplot} illustrates the joint \pdf{}s of all \pmodel{}. The \pdf{} is a multivariate Gaussian that projects into Gaussians when considering the marginalized \pdf{} for all the individual \pmodel{}. Figure \ref{fig:sqrtNiso} then demonstrates that the \pdf{} width indeed scales as $1/\sqrt{N_{*}}$. Figure \ref{fig:isoSph_CLT} illustrates even more that \RM{} behaves like an unbiased maximum likelihood estimator: The average parameter estimates from many mock samples with identical underlying \pmodel{} are very close to the input \pmodel{}, and the distribution of the actual parameter estimates are a Gaussian around it. 