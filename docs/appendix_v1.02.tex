%FIGURE: isoSphFlexIncompR and isoSphFlexIncompZ in mock data space

\begin{figure}
\plotone{figs/isoSphFlexIncompZ_mockdata.eps}
\caption{[TO DO: Rewrite Caption] Selection function and mock data distribution for investigating radial (Example 1 \& 2, left) and vertical (Example 3 \& 4, right) incompleteness of the data. The survey volume is a sphere around the sun with $r_\text{max} = 3$ kpc. In Example 1 \& 2 (Example 3 \& 4) the percentage of observed stars is decreasing linearly with radius from the sun (height above the Galactic plane), as demonstrated in the first row of panels. How fast this detection rate drops is quantized by the factor $\epsilon_r$ ($\epsilon_z$) in eq. (\ref{eq:radial_incomp}) (eq. (\ref{eq:vertical_incomp})). Different mock data sets have different $\epsilon_r$ ($\epsilon_z$). Histograms for four data sets, each with 20,000 and drawn from two \MAPs ("hot" in red and "cool" in blue, see table \ref{tbl:referenceMAPs}) and with two different $\epsilon_r$ ($\epsilon_z$), 0 and 0.7, are shown in the lower two panels for illustration purposes.[TO DO: Re-do, if new analyses are in violin plot.]} 
\label{fig:incompZ_mockdata}
\end{figure}

%FIGURE: isoSphFlexIncompZ

\begin{figure}
\plotone{figs/isoSphFlexIncompZ_violins.eps}
\caption{Caption [TO DO] (This was done using the current qDF to set the fitting range. Nvelocity=24 and Nsigma=5 is high enough, though there is one analyse (hot, no3) for which it did not work. Maybe redo with fiducial qD. [TO DO] Also compare these results with the result when marginalizing over $vT$ [TO DO])} 
\label{fig:isoSphFlexIncompZ_violins}
\end{figure}

\begin{figure}
\plotone{figs/isoSphFlexIncomp_marginal_violins.eps}
\caption{Caption [TO DO] ([TO DO: Redo all analyses for which MCMC did not converge to expected peak, and for which b <0 was not excluded. ???])} 
\label{fig:isoSphFlexIncompZ_violins}
\end{figure}