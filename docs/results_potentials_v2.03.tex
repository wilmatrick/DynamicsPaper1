\subsection{The Implications of a Gravitational Potential not from the Space of Model Potentials} \label{sec:results_potential}

%Motivation for the Test

We now explore what happens when the mock data were drawn from one axisymmetric potential family, here \texttt{MW14-Pot}, and is then modelled considering potentials from only another axisymmetric family, here \texttt{KKS-Pot} (compare the second and fourth panel in Figure \ref{fig:ref_pots}). In the analysis we assume the circular velocity at the Sun to be fixed and known \HW{[TO DO: Comment from Hans-Walter: Do we have reason to believe that this very restrictive assumption does not qualitatively impact our upshot (quantitative differences are OK).]} and only fit the parametric potential form. The results are shown in Figure \ref{fig:MW14vsKKS2SphFlex}.




%Results on the potential

The reference potential parameters \HW{[TO DO: Comment from HW: What does "reference" mean here exactly? Is this an independent exercise, to ask which parameters are expected when fitting potential to potential? (I don't know, what he means...)]} of the \texttt{KKS-Pot} in Table \ref{tbl:referencepotentials} were found by adjusting the 2-component Kuzmin-Kutuzov St\"{a}ckel potential by \citet{bat94} such that it generates radial and vertical force profiles similar to the \texttt{MW14-Pot} from \citet{bov15} (dotted gray lines in Figure \ref{fig:MW14vsKKS2SphFlex}). We then run \RM{} using these ``inconsistent'' families of gravitational potentials, and find a good fit to the data in configuration space (see Figure \ref{fig:MW14vsKKS2SphFlex_mockdata_residuals}). The results from \RM{} analysis for the potential shown in Figure \ref{fig:MW14vsKKS2SphFlex}, red for a \texttt{hot} mock data \MAP{} and blue for a \texttt{cool} \MAP{}, give an comparable good or even better agreement with the true potential than the (by-eye) fit directly to the potential: especially the force contours, to which the orbits are sensitive, and the rotation curve are very tightly constrained and reproduce the true potential even outside of the observed volume of the mock tracers. This demonstrates that \RM{} fitting inferres a potential that in its actual properties resembles the input potential for the mock data as closely as possible, given the differences in functional forms.
\\The density contours are less tightly constrained than the forces, but we still capture the essentials: the \texttt{hot} \MAP{} from Table \ref{tbl:referenceMAPs} constrains the halo; especially at smaller radii it is equally good or better than the \texttt{cool} \MAP{}. The \texttt{cool} \MAP{} gives tighter constraints on the halo in the outer region and recovers the disk better than the \texttt{hot} \MAP{}. This is in concordance with expectations as the \texttt{cool} \MAP{} has a longer tracer scale length and is more confined to the disk than the \texttt{hot} \MAP{} and therefore also probes the Galaxy in these regions better.
\\Overall the best fit disk is less dense in the midplane than the true disk. 

%Results on the qDF

Figure \ref{fig:MW14vsKKS2SphFlex_violins} compares the true qDF parameters with the best fit parameters for this case. While tracer scale length and radial velocity dispersion profile are very well recovered, we misjudge the radial profile of the vertical velocity dispersion as $\sigma_{0,z}$ and $h_{\sigma,z}$ are both underestimated, which leads to a steeper profile and a lower dispersion around the Sun.






%====================================================================

\begin{figure*}
\plotone{figs/MW14vsKKS2SphFlex_mockdata_residuals.eps}
\caption{Comparison of the distribution of mock data in configuration space created in the \texttt{MW14-Pot} potential (solid lines) with a \texttt{hot} (red) and \texttt{cool} (blue) \MAP{} (Test \ref{test:MW14vsKKS2SphFlex} in Table \ref{tbl:tests}), and the best fit distribution using a \texttt{KKS-Pot} potential (dashed lines). The best fit potentials are shown in Figure \ref{fig:MW14vsKKS2SphFlex} and the corresponding best fit qDF parameters in Figure \ref{fig:MW14vsKKS2SphFlex_violins}. The best fit [TO DO: Continue Caption] [TO DO: Potential and/or population names in typewriter font]}
\label{fig:MW14vsKKS2SphFlex_mockdata_residuals}
\end{figure*}


\begin{figure*}
\plotone{figs/MW14vsKKS2SphFlex_contours_compare.eps}
\caption{Recovery of the gravitational potential if the assumed potential model (\texttt{KKS-Pot} with fixed $v_\text{circ}(R_\odot)$) and the true potential of the (mock) stars (\texttt{MW14-Pot} in Table \ref{tbl:referencepotentials}) is slightly different. We show the circular velocity curve, as well as contours of equal density, radial and vertical force in the $R$-$z$-plane, and compare the true potential with 50 [TO DO: CHECK] sample potentials drawn from the posterior distribution function found with the MCMC for a \texttt{hot} (red) and a \texttt{cool} \MAP{} (blue). All model parameters are given as Test \ref{test:MW14vsKKS2SphFlex} in Table \ref{tbl:tests}. \Wilma{[TO DO: Do more analyses???] [TO DO: fancybox Legend] [TO DO: Potential and/or population names in typewriter font] [TO DO: Reference correct Table in Plot] [TO DO: Redo whole analysis with vcirc not being fixed (HW is not sure if this really doesn't make a difference.]}}
\label{fig:MW14vsKKS2SphFlex}
\end{figure*}

\begin{figure*}
\plotone{figs/MW14vsKKS2SphFlex_violins.eps}
\caption{Recovery of the qDF parameters for the case where the true and assumed potential deviate from each other (Test \ref{test:MW14vsKKS2SphFlex} in Table \ref{tbl:tests}). The thick red (blue) lines represent the true qDF parameters of the \texttt{hot} (\texttt{cool}) qDF in Table \ref{tbl:referenceMAPs} used to create the mock data, surrounded by a 10\% error region. The grey violins are the marginalized likelihoods for the qDF parameters found simultaneously with the potential constraints shown in Figure \ref{fig:MW14vsKKS2SphFlex}. [TO DO: rename $h_{\sigma R}$ to $h_{\sigma,R}$, $\sigma_R$ to $\sigma_{R,0}$ and analogous for $z$]}
\label{fig:MW14vsKKS2SphFlex_violins}
\end{figure*}


%====================================================================




