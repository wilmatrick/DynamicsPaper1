\section{Stuff that still needs to be done or thought about}

\paragraph{Questions that haven't been covered so far:}
\begin{itemize}
\item What happens, when the errors are not uniform?
\item What if errors in distance matter for selection?
\end{itemize}

\paragraph{Stuff that needs to be further examined in fig. \ref{fig:wedFlexVol_bias_vs_SE} about the survey volume:}
\begin{itemize}
\item[[TO DO]] Maybe add volume at smaller radius with large vertical extent?
\item[[TO DO]] Do we explicitely want to test, if it matters, if the radial coverage is larger or smaller the disk scale length, and the vertical coverage is larger or smaller than the disk scale height?
\end{itemize}

\paragraph{Stuff that needs to be further examined about the robustness against data incompleteness:}
\begin{itemize}
\item[[TO DO]] Make similar test as isoSphFlexIncompR, but with KKS potential, to test, if this
robustness is a special case for the isochrone potential.
\end{itemize}

\paragraph{General Stuff}
\begin{itemize}
\item[[TO DO:]] Look up what McMillan \& Binney 2013 have to say about the numerical accuracy of the normalisation. Sanders \& Binney (2015) are quoting them on that matter.
\item[[TO DO:]] einheitlich, convolution of model probability with measurement errors, nicht deconvolution of likelihood
\item[[TO DO:]] For all Binney \& Tremaine 2008 references include the corresponding section.
\item[[TO DO:]] Make sure that probabilities have lowercase $p$ everywhere.
\item[[TO DO:]] Make sure there are no Tables overlapping with the page numbers.
\end{itemize}

%=====================================================

%15. May 2015 --> Meet with HW to write
%
%* 20,000 stars --> much larger than current sample sizes (200 stars), but forecasting to larger sample sizes with Gaia (?)
%* Fig. 3,4,5 --> Behaviour in the limit of large samples
%* Change order of sections according to order in Results section intro
%* mention in introduction that we do not investigate axisymmetry
%* limit of lousy data --> model assumptions are not limiting. very good data --> model assumptions are limiting. Bovy & Rix: 150 stars per MAP. When we get larger samples sizes, the modelling will be limited by the model assumptions.
%* in section on numerical accuracy erwähnen, dass wir in the limit of many stars aufpassen müssen, dass wir die normalisierung genau genug berechnen.
%* Check how many stars were typically in a Bovy&Rix13 MAP
%* Priors: We want parameter estimates that alre tight enough, such that it does not matter, if we had assumed a flat or a logarithmically flat prior
%* Introduce somewhere the 20,000 stars 
%* change <_ 1 in eq. 5 to <_ 1/N_sample
%* Mach einheitlich: width of pdf, likelihood, Standard error --> $\sigma_p$ ???
%* schwarze punkte in (un)-bias CLT plot: call "pdf expectation value"
%* Error on the width of the likelihood scales also with 1/sqrt(N) or sqrt(N-1) --> nicht in sqrtN figure einzeichnen, weil mein scatter größer ist. Neu berechnen?
%* obsvolumetest: orange volume eine breite weiter nach oben, also leicht oberhalb der plane.
%* Check that CLT plot and volumetestplot have both 20,000 stars

