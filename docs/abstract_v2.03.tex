\begin{abstract}
We present \RM{}, a dynamical modelling machinery that aims to recover the Milky Way's (MW) gravitational potential and the orbit distribution of stellar populations in the Galactic disk. \RM{} is a full likelihood analysis that models the observed positions and velocities of stars with an equilibrium, three-integral distribution function (DF) in an axisymmetric potential. In preparation for the application to the large data sets of modern surveys like Gaia, we create and analyze a large suite of mock data sets and develop qualitative ``rules of thumb'' for which characteristics and limitations of data, model and machinery affect constraints on the potential and DF most. We find that, while the precision of the recovery increases with the number of stars, the numerical accuracy of the likelihood normalisation becomes increasingly important and dominates the computational efforts. The modelling has to account for the survey's selection function, but \RM{} seems to be very robust against small misjudgments of  the data completeness. Large radial and vertical coverage of the survey volume gives in general the tightest constraints. But no observation volume of special shape or position and stellar population should be clearly preferred, as there seem to be no stars that are on manifestly more diagnostic orbits. We propose a simple approximation to include measurement errors at comparably low computational cost that works well if the distance error is $\lesssim 10\%$. The model parameter recovery is also still possible, if the proper motion errors are known to within 10\% and are $\lesssim 2 \text{ mas yr}^{-1}$. We also investigate how small deviations of the stars' distribution from the assumed DF influence the modelling: An over-abundance of high velocity stars affects the potential recovery more strongly than an under-estimation of the DF's low-velocity domain. Selecting stellar populations according to mono-abundance bins of finite size can give reliable modelling results, as long as the DF parameters of two neighbouring bins do not vary more than 20\% [TO DO: CKECK]. As the modelling has to assume a parametric form for the gravitational potential, deviations from the true potential have to be expected. We find, that in the axisymmetric case we can still hope to find a potential that is indeed a reliable best fit within the limitations of the assumed potential. Overall \RM{} works as a reliable and unbiased estimator, and is robust against small deviations between model and the ``real world''. 
\end{abstract}





