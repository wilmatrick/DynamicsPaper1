\subsection{What if our assumed potential model differs from the real potential?} \label{sec:results_potential}

%Motivation for the Test

In the long run \RM should incorporate a family of gravitational potential models that is flexible enough to reproduce the essential features of the MW's true mass distribution. Here we inspect if we can already give constraints on the true potential, even if our assumed potential is still too rigid, either because of a low number of free potential parameters, or because our beliefs about the overall parametric form of the MW's potential are slightly wrong. While our fundamental assumption of the potential's axisymmetry springs immediately to mind, being at odds with the obvious existence of a bar and spiral arms in the MW, we will not dive into investigating the implications in the scope of this paper. We rather focus on the case where the mock data was drawn from one axisymmetric potential ("MW14-Pot") and is then analysed using another axisymmetric potential family ("KKS-Pot"), that does \emph{not} incorporate the true potential (compare the second and fourth panel in Fig. \ref{fig:ref_pots}). In the analysis we assume the circular velocity at the sun to be fixed and known and only fit the parametric potential form. The results are shown in Fig. \ref{fig:MW14vsKKS2SphFlex}.




%Results on the potential

The reference potential parameters of the "KKS-Pot" in Table \ref{tbl:referencepotentials} were found by adjusting the 2-component Kuzmin-Kutuzov St\"{a}ckel potential by \citet{bat94} such that it generates radial and vertical force profiles similar to the "MW14-Pot" from \citet{bov15} (dotted gray lines in Fig. \ref{fig:MW14vsKKS2SphFlex}). The analysis results from \RM shown in Fig. \ref{fig:MW14vsKKS2SphFlex}, red for a "hot" mock data \MAP and blue for a "cool" \MAP, give an comparable good or even better agreement with the true potential than the (by-eye) fit directly to the potential: especially the force contours, to which the orbits are sensitive to, and the rotation curve are very tightly constrained and reproduce the true potential even outside of the observed volume of the mock tracers. This demonstrates that \RM provides an optimal best fit potential within the capabilities of the parametric potential model.
\\The density contours are less tightly constrained than the forces, but we still capture the essentials: The "hot" \MAP from Table \ref{tbl:referenceMAPs} constrains the halo, especially at smaller radii it is equally good or better than the "cool" \MAP. The "cool" \MAP gives tighter constraints on the halo in the outer region and recovers the disk better than the "hot" \MAP. This is in concordance with expectations as the "cool" \MAP has a longer tracer scale length and is more confined to the disk than the "hot" \MAP and therefore also probes the Galaxy in these regions better.

%Results on the qDF

[TO DO: Results on the qDF]

[TO DO: Plot showing the true and best fit distribution]






%====================================================================

\begin{figure}
\plotone{figs/MW14vsKKS2SphFlex_contours_compare.eps}
\caption{Recovery of the gravitational potential if the assumed potential model ("KKS-Pot" with fixed $v_\text{circ}(R_\odot)$) and the true potential of the (mock) stars ("MW14-Pot" in Table \ref{tab:referencepotentials}) is slightly different. We show the circular velocity curve, as well as contours of equal density, radial and vertical force in the $R$-$z$-plane, and compare the true potential with 50 [TO DO: CHECK] sample potentials drawn from the posterior distribution function found with the MCMC for a "hot" (red) and a "cool" \MAP (blue). All model parameters are given as Test \textcircled{8} in Table \ref{tab:tests}. [TO DO: Do more analyses???]}
\label{fig:MW14vsKKS2SphFlex}
\end{figure}

\begin{figure}
\plotone{figs/MW14vsKKS2SphFlex_violins.eps}
\caption{Recovery of the qDF parameters for the case where the true and assumed potential deviate from each other (Test \textcircled{8} in Table \ref{tbl:tests}). The thick red (blue) lines represent the true qDF parameters of the "hot" ("cool") qDF in Table \ref{tbl:referenceMAPs} used to create the mock data, surrounded by a 10\% error region. The grey violins are the marginalized likelihoods for the qDF parameters found simultaneously with the potential constraints shown in Fig. \ref{fig:MW14vsKKS2SphFlex}.}
\label{fig:MW14vsKKS2SphFlex_violins}
\end{figure}


%====================================================================




