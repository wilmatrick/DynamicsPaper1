\subsection{The implications of a gravitational potential not from the space of model potentials} \label{sec:results_potential}

We now explore what happens when the mock data were drawn from one axisymmetric potential family, here \texttt{MW14-Pot}, and is then modelled considering potentials from only another axisymmetric family, here \texttt{KKS-Pot} (see Table \ref{tbl:referencepotentials} and Figure \ref{fig:ref_pots}). In the analysis we assume the circular velocity at the Sun to be fixed and known and only fit the parametric potential form.\footnote{We made sure that $v_\text{circ}(R_\odot)$ can be very well recovered when included in the fit of a \texttt{cool} population. The model assumption that $v_\text{circ}(R_\odot)$ is known does therefore not affect the discussion qualitatively.}

We analyse a mock data set from each a \texttt{hot} and \texttt{cool} stellar population (see Test \ref{test:MW14vsKKS2SphFlex} in Table \ref{tbl:tests}) with high numerical accuracy. The distributions generated from the best fit parameters reproduce the data in configuration space very well (see Figure \ref{fig:MW14vsKKS2SphFlex_mockdata_residuals}).

The results for the potential are shown in Figure \ref{fig:MW14vsKKS2SphFlex}. We find that the potential recovered by \RM{} is in good agreement with the true potential. Especially the force contours, to which the orbits are sensitive, and the rotation curve are very tightly constrained and reproduce the true potential even outside of the observed volume of the mock tracers.

Overplotted in Figure \ref{fig:MW14vsKKS2SphFlex} is also the \texttt{KKS-Pot} with the parameters from Table \ref{tbl:referencepotentials}, which were fixed based on a (by-eye) fit \emph{directly} to the force field (within $r_\text{max}=4~\text{kpc}$ from the Sun) and rotation curve of the \texttt{MW14-Pot}. The potential found with the \RM{} analysis is an equally good or even slightly better fit. This demonstrates that \RM{} fitting infers a potential that in its actual properties resembles the input potential for the mock data as closely as possible, given the differences in functional forms.

The density contours are less tightly constrained than the forces, but we still capture the essentials. Overall the best fit disk is less dense in the midplane than the true disk, because the generation of very flattened components like exponential disks with St\"{a}ckel potentials is very difficult.

Figure \ref{fig:MW14vsKKS2SphFlex_violins} compares the true qDF parameters with the best fit qDF parameters belonging to the best fit potentials from Figure \ref{fig:MW14vsKKS2SphFlex}. While we recover $h_R$, $\sigma_{R,0}$ and $h_{\sigma,R}$ within the errors, we misjudge the parameters of the vertical velocity dispersion ($\sigma_{0,z}$ and $h_{\sigma,z}$), even though the actual mock data distribution is well reproduced. This discrepancy could be connected to the \texttt{KKS-Pot} not being able to reproduce the flattness of the disk. Also, $\sigma_z$ and $\sigma_R$ in Equations \ref{eq:sigmaRRg}-\ref{eq:sigmazRg} are scaling profiles for the qDF (cf. BR13) and how close they are to the actual velocity profile depends on the choice of potential.










