\section{Dynamical Modelling}

\subsection{Actions and Potential Models}  \label{sec:potentials}

\paragraph{Actions.} Orbits in axisymmetric potentials are best described and fully specified by the three actions $J_R, J_z$ and $J_\phi=L_z$. They are integrals of motion and generally defined as
\begin{equation}
J_i = \frac{1}{2\pi} \int_\text{orbit} p_i(t) \diff x_i(t) \label{eq:action_general}
\end{equation}
and depend on the potential via the connection between position $x_i$ and momentum $p_i$ along the orbit. Actions have a clear physical meaning: They quantify the amount of oscillation in each coordinate direction of the full orbit [REF]. The position of a star along the orbit is denoted by a set of angles, which form together with the angles a set of canonical conjugate phase-space coordinates \citep{bin08}. Even though actions are the optimal choice as orbit labels and arguments for stellar distribution functions, their computation is very expensive.

\paragraph{Action calculation.} The action calculation depends on the choice of potential in which the star moves: The spherical isochrone \citep{bin08} is the only potential for which Eq. (\ref{eq:action_general}) takes an analytic form. For axisymmetric St\"{a}ckel potentials actions can be calculated exactly by the (numerical) evaluation of a single integral. In all other potentials numerically calculated actions will always be approximations, unless Eq. (\ref{eq:action_general}) is integrated up to infinity.  A computational fast way to get actions for arbitrary axisymmetric potentials is the "St\"{a}ckel fudge" by \citet{bin12}, which locally approximates the potential by a St\"{a}ckel potential. To speed up the calculation even more, an interpolation grid for $J_R$ and $J_z$ in energy $E$, angular momentum $L_z$ and [TO DO: what else???] can be build out of these St\"{a}ckel fudge actions, as described in \citet{bov15}.\footnote{[TO DO: Write which numerical accuracy I needed for the grid, as the default values were not good enough.]} 

\paragraph{Potential models.} In our modelling we assume a family of parametrized potentials with a fixed number of free parameters. We use different kinds of potentials: Besides the Milky Way like potential from \citet{bov13} ("MW13-Pot" ) with bulge, disk and halo, we also extensively use the spherical isochrone potential ("Iso-Pot") in our test suites to make use of the analytic (and therefore exact and fast) way  to calculate actions. In addition we use the 2-component Kuzmin-Kutuzov St\"{a}ckel potential by \citet{bat94} ("KKS-Pot"), which displays a disk and halo structure and also provides exact actions. Table \ref{tbl:referencepotentials} summarizes all reference potentials together used in this work with their free parameters $p_\Phi$. The density distribution of these potentials is illustrated in Fig. \ref{fig:ref_pots}.

%======================================================================================



%=============================================

%FIGURE: reference potentials

\begin{figure*}
\includegraphics[width=\textwidth]{figs/reference_potentials.eps}
\caption{Density distribution of the four reference galaxy potentials in Table \ref{tbl:referencepotentials}, for illustration purposes. These potentials are used throughout this work for mock data creation and potential recovery. [TO DO: Halo sichtbarer machen, evtl. mit isodensity contours]}
\label{fig:ref_pots}
\end{figure*}

%=============================================

\subsection{Distribution Function} \label{sec:qDF}

\paragraph{Distribution Function.} Motivated by the findings of Bovy et al. 2012??? and \citet{tin13} about the simple phase-space structure of \MAPs, and following \citet{bov13} and their successful application, we also assume that each \MAP follows a single qDF of the form given by \citet{bin11}.  This qDF  is a function of the actions $\vect{J}=(J_R,J_z,L_z)$ and has the form
\begin{eqnarray}
\text{qDF}(\vect{J} \mid p_\text{DF}) &=& f_{\sigma_R}\left(J_R,L_z \mid p_\text{DF}\right) \nonumber \\ \times f_{\sigma_z}\left(J_z,L_z \mid p_\text{DF}\right)\label{eq:df_general}\\
\text{with}\hspace{2cm}&\nonumber\\
f_{\sigma_R}\left(J_R,L_z \mid p_\text{DF}\right) &=& n \times \frac{\Omega}{\pi\sigma_R^2(R_g) \kappa}\exp\left(-\frac{\kappa J_R}{\sigma_R^2(R_g)} \right) \nonumber\\
&& \times \left[1+\tanh\left(L_z/L_0\right) \right]\\
f_{\sigma_z}\left(J_z,L_z \mid p_\text{DF} \right) &=& \frac{\nu}{2 \pi \sigma_z^2(R_g)} \exp\left( -\frac{\nu J_z}{\sigma_z^2(R_g)} \right) \\
\end{eqnarray}
Here $R_g \equiv R_g(L_z)$ and $\Omega\equiv \Omega(L_z)$ are the (guidig-center) radius and the circular frequency of the circular orbit with angular momentum $L_z$ in a given potential. $\kappa\equiv \kappa(L_z)$ and $\nu\equiv \nu(L_z)$ are the radial/epicycle ($\kappa$) and vertical ($\nu$) frequencies with which the star would oscillate around the circular orbit in $R$- and $z$-direction when slightly perturbed \citep{bin08}. The term $\left[1+\tanh\left(L_z/L_0\right) \right]$ suppresses counter-rotation for orbits in the disk with $L \gg L_0$ which we set to a random small value ($L_0 = 10 \times R_\odot/8 \times v_\text{circ}(R_\odot)/220$).
\\For this qDF to be able to incorporate the findings by Bovy et al. 2012??? about the phase-space structure of \MAPs summarized in \S\ref{sec:intro}, we set the functions $n$,  $\sigma_R$ and $\sigma_z$, which indirectly set the stellar number density and radial and vertical velocity dispersion profiles,
\begin{eqnarray}
n(R_g \mid p_\text{DF}) &\propto& \exp\left(-\frac{R_g}{h_R} \right)\\
\sigma_R(R_g \mid p_\text{DF}) &=& \sigma_{R,0} \times \exp\left(- \frac{R_g-R_\odot}{h_{\sigma_R}} \right)\label{eq:sigmaRRg}\\
\sigma_z(R_g \mid p_\text{DF}) &=& \sigma_{z,0} \times \exp\left(- \frac{R_g-R_\odot}{h_{\sigma_z}} \right)\label{eq:sigmazRg}.
\end{eqnarray}
The qDF for each \MAP has therefore a set of five free parameters $p_\text{DF}$: the density scale length of the tracers $h_R$, the radial and vertical velocity dispersion at the solar position $R_\odot$, $\sigma_R,0$ and $\sigma_z,0$, and the scale lengths $h_{\sigma_R}$ and $h_{\sigma_z}$, that describe the radial decrease of the velocity dispersion. The \MAPs we use for illustration through out this work are summarized in Table \ref{tbl:referenceMAPs}.

\paragraph{Tracer Density.} One crucial point in our dynamical modelling technique (\S ???), as well as in creating mock data (\S\ref{sec:mockdata}), is to calculate the (axisymmetric) spatial tracer density $\rho_\text{DF}(\vect{x} \mid p_{\Phi},p_\text{DF})$ for a given qDF and potential . We do this by integrating the qDF at a given $(R,z)$ over all three velocity components, using a $N_\text{velocity}$-th order Gauss-Legendre quadrature for each integral:
\begin{eqnarray}
&&\rho_\text{DF}(R,|z| \mid p_{\Phi},p_\text{DF}) \nonumber\\
&&= \int_{-\infty}^{\infty} \text{qDF}(\vect{J}[R,z,\vect{v} \mid p_{\Phi}] \mid p_\text{DF}) \Diff3\vect{v}  \label{eq:tracerdensity_general}\\
&&\approx \int_{-N_\text{sigma} \sigma_R(R \mid p_\text{DF})}^{N_\text{sigma} \sigma_R(R \mid p_\text{DF})} \int_{-N_\text{sigma}\sigma_z(R \mid p_\text{DF})}^{N_\text{sigma} \sigma_z(R \mid p_\text{DF})} \int_{0}^{1.5 v_\text{circ}(R_\odot)}  \nonumber\\
& & \hspace{1cm} \text{qDF}(J[R,z,\vect{v} \mid p_{\Phi}] \mid p_\text{DF}) \diff v_T \diff v_z \diff v_R, \label{eq:tracerdensity}
\end{eqnarray}
where $\sigma_R(R \mid p_\text{DF})$ and $\sigma_z(R \mid p_\text{DF})$ are given by eq. (\ref{eq:sigmaRRg}) and (\ref{eq:sigmazRg}) and the integration ranges are motivated by Fig. \ref{fig:mockdatadistr}. For a given $p_\Phi$ and $p_\text{DF}$ we explicitly calculate the density on $N_\text{spatial} \times N_\text{spatial}$ regular grid points in the $(R,z)$ plane; in between grid points the density is evaluated with a bivariate spline interpolation. The grid is chosen to cover the extent of the observations for $z>0$. The total number of actions that need to be calculated to set up the density interpolation grid is $N_\text{spatial}^2 \cdot N_\text{velocity}^3$. Fig. ??? shows the importance of choosing $N_\text{spatial}$, $N_\text{velocity}$ and $N_\text{sigma}$ sufficiently large in order to get the density with an acceptable numerical accuracy. 

%======================================================================================


%======================================================================================


\subsection{Selection Function} \label{sec:selectionfunction}

\paragraph{Galactic Coordinate System.} Our modelling takes place in the Galactocentric rest-frame with cylindrical coordinates $\vect{x} \equiv (R,\phi,z)$ and corresponding velocity components $\vect{v} \equiv (v_R,v_\phi,v_z)$. If the stellar phase-space data is given in observed coordinates, position $\tilde{\vect{x}} \equiv(\alpha,\delta,m-M)$ in right ascension $\alpha$, declination $\delta$ and distance modulus $(m-M)$, and velocity $\tilde{\vect{v}} \equiv (\mu_\alpha,\mu_\delta,v_\text{los})$ as proper motions $\vect{\mu}=(\mu_\alpha,\mu_\delta)$ [TO DO: cos somwhere???] and line-of-sight velocity $v_\text{los}$, the data $(\tilde{\vect{x}},\tilde{\vect{v}})$ has to be converted first into the Galactocentric rest-frame coordinates $(\vect{x},\vect{v})$ using the sun's position and velocity. For simplicity we assume for the sun
\begin{eqnarray*}
(R_\odot,\phi_\odot,z_\odot) &=&(8 \text{ kpc}, 0^\circ, 0 \text{ kpc})\\
(v_{R,\odot},v_{T,\odot},v_{z,\odot}) &=& (0,230,0) \text{ km s}^{-1}.
\end{eqnarray*}

\paragraph{Selection Function.} A survey's selection function can be understood as a subvolume in the space of observables: e.g. position on the plane of the sky (limited by the pointing of the survey), distance from the sun (limited by the brightness of the stars and the sensitivity of the detector), colors and metallicity of the stars (limited by survey mode and targeting).
\\Within the framework of this paper, using only mock data for testing purposes, we ignore target cuts in colors and metallicity and simply use spatial selection functions, which we define as
\begin{eqnarray*}
\text{sf}(\vect{x}) \equiv \begin{cases}
\text{completeness}(\vect{x}) &\text{if $\vect{x}$ within observed volume}\\
0 & \text{outside}
\end{cases}
\end{eqnarray*}
It's value describes the probability to observe a star at $\vect{x}$. 
\\For the observed volume we use simple geometrical shapes: Either a sphere of radius $r_\text{max}$ with the sun at its center, or a "wedge", which we define as the angular segment of an cylindrical annuli, i.e. the volume with $R \in [R_\text{min},R_\text{max}],\phi \in [\phi_\text{min},\phi_\text{max}],z \in [z_\text{min},z_\text{max}]$ within the model galaxy. The sharp outer cut of the survey volume could be understood as the detection limit in apparent brightness in the case, where all stars have the same luminosity.
\\The completeness is, in our framework, a function of position with $0 \leq \text{completeness}(\vect{x}) \leq 1$ everywhere inside the observed volume. It could be understood as a position-dependent detection probability. Unless explicitly stated otherwise, we use everywhere
$$\text{completeness}(\vect{x}) = 1.$$