\subsection{Impact of Misjudging the Completeness of the Data Set} \label{sec:results_incompR}

The completeness function (see Section\ref{sec:selectionfunction}) depends on the characteristics and mode of the survey. It can be very complex and is therefore sometimes not perfectly known. We investigate how much the recovery of the potential can be affected by imperfect knowledge of the selection function. We do this by creating mock data with varying incompleteness (within a maximal survey volume), while assuming constant completeness in the analysis. The mock data comes from a sphere around the sun with an incompleteness function that drops linearly with distance $r$ from the sun (see Test \ref{test:isoSphFlexIncomp}, Example 1, in Table \ref{tbl:tests} and Figure \ref{fig:isoSphFlexIncompR_mockdata}). This captures the relevant case of stars being less likely to be observed (than assumed) the further away they are (e.g. due to unknown dust obscuration). We demonstrate that the potential recovery with \RM{} is very robust against somewhat wrong assumptions about the radial completeness of the data (see Figure \ref{fig:isoSphFlexIncompR_violins}). Apparently, much information about the potential comes from the rotation curve measurements in the plane, which is not affected by the incompleteness of the sample. In Appendix \ref{sec:incompZ} we also show that the robustness is somewhat less striking but still persists for small misjudgments of the incompleteness in vertical direction, parallel to the disk plane (Figures \ref{fig:isoSphFlexIncompZ_mockdata} and \ref{fig:isoSphFlexIncompZ_violins}). This could model the effect of wrong corrections for interstellar extinction in the plane. We also investigate in Appendix \ref{sec:incompZ} if indeed most of the information is stored in the rotation curve \HW{[TO DO: Comment by HW: I don't have an immediate solution for this, but again, it seems the interesting question of "how much of the information is in the rotation curve" is 'hidden' in the section on selection functions...]}. For this we use the same mock data sets as analysed in Figures \ref{fig:isoSphFlexIncompR_violins} and \ref{fig:isoSphFlexIncompZ_violins}, but without including the tangential velocities in the modelling (by marginalizing the likelihood over $v_T$). In this case the potential is much less tightly constrained, even for 20,000 stars. For only small deviations of true and assumed completeness ($\lesssim 10\%$) we can however still incorporate the true potential in our fitting result (see Figure \ref{fig:isoSphFlexIncomp_marginal_violins}). 



%FIGURE: isoSphFlexIncompR in mock data space

\begin{figure}
\includegraphics[width=\columnwidth]{figs/isoSphFlexIncompR_mockdata.eps}
\caption{Selection function and mock data distribution for investigating radial incompleteness of the data. All model parameters are summarized as Test \ref{test:isoSphFlexIncomp}, Example 1, in Table \ref{tbl:tests}. The survey volume is a sphere around the sun and the percentage of observed stars is decreasing linearly with radius from the sun, as demonstrated in the left panel. How fast this detection/incompleteness rate drops is quantified by the factor $\epsilon_r$. Histograms for four data sets, drawn from two \MAPs{} (\texttt{hot} in red and \texttt{cool} in blue, see Table \ref{tbl:referenceMAPs}) and with two different $\epsilon_r$, 0 and 0.7, are shown in the right panel for illustration purposes. [TO DO: Potential and/or population names in typewriter font]} 
\label{fig:isoSphFlexIncompR_mockdata}
\end{figure}

%FIGURE: isoSphFlexIncompR

\begin{figure}
\centering
\plotone{figs/isoSphFlexIncompR_violins_2.eps}
\caption{Influence of wrong assumptions about the radial incompleteness of the data on the parameter recovery with \RM{}. Each mock data set was created with different incompleteness parameters $\epsilon_r$ (shown on the $x$-axis and illustrated in Figure \ref{fig:isoSphFlexIncompR_mockdata}) and the model parameters are given as Test \ref{test:isoSphFlexIncomp}, Example 1, in Table \ref{tbl:tests}. The analysis however did not know about the incompleteness and assumed that all data sets had constant completeness within the survey volume ($\epsilon_r = 0$). The marginalized likelihoods from the fits are shown as violins. The green lines mark the true potential parameters (\texttt{Iso-Pot}) and the red and blue lines the true qDF parameters (\texttt{hot} \MAP in red and \texttt{cool} \MAP in blue), which we tried to recover. The \RM{} method seems to be very robust against small to intermediate deviations between the true and the assumed data incompleteness. \Jo{[TO DO: Jo suggested to also remove the $h_R$ panel, but I like, that one can see that it is the spatial tracer distribution that drives the little degradation of the recovery.]}} 
\label{fig:isoSphFlexIncompR_violins}
\end{figure}

\Wilma{[TO DO: Mention in text or caption how the panels looked that I removed.]}