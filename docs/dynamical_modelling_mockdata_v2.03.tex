%%====================================================================
%
%%FIGURE: distribution of mock data in action and configuration space
%
%\begin{figure*}
%\plotone{figs/kks2WedgeEx_mockdata.eps}
%\caption{Distribution of mock data in action space (2D iso-density contours, enclosing 80\% of the stars, the two central and the lower left panel) and configuration space (1D histograms, right panels), depending on shape and position of the survey observation volume and temperature of the stellar population. The parameters of the mock data model is given as Test \ref{test:kks2WedgeEx} in Table \ref{tbl:tests}. In the upper left panel we demonstrate the shape of the two different \texttt{wedge}-like observation volumes within which we were creating each a \texttt{hot} (red) and \texttt{cool} (blue) mock data set: a large volume centred on the Galactic plane (solid lines) and a smaller one above the plane (dashed lines). The distribution in action space visualizes how orbits with different actions also reach into different regions within the Galaxy. The 1D histograms on the right illustrate that qDFs generate realistic stellar distributions in galactocentric coordinates $(R,z,\phi,v_R,v_z,vT)$. \Wilma{[TO DO: fancybox Legend] [TO DO: Potential and/or population names in typewriter font]} \Jo{[TO DO: Jo suggests to make two or three separate figures out of this. I'm not yet convinced, as I think it is nice and tidy like this.]}} 
%\label{fig:kks2WedgeEx}
%\end{figure*}
%
%
%%===========================================================================================================================================================================================


\subsection{Mock Data} \label{sec:mockdata}

We will rely on mock data as input to explore the limitations of the modelling. We investigate this, we assume first that our measured stars do indeed come from our assumed families of potentials and distribution functions and draw mock data from a given true distribution. Subsequently, we manipulate and modify these mock data sets to mimic observational effects.\\
The distribution function is given in terms of actions and angles. The transformation $(\vect{J}_i,\vect{\theta}_i) \longrightarrow (\vect{x}_i,\vect{v}_i)$ is however difficult to perform and computationally much more expensive than the transformation $(\vect{x}_i,\vect{v}_i) \longrightarrow (\vect{J}_i,\vect{\theta}_i)$. We employ a fast and simple two-step method for drawing mock data from an action distribution function, which also accounts effectively for a given survey selection function.\\

In the first step we draw positions $\vect{x}_i$ for our mock data stars from the selection function and tracer density. We start by setting up the interpolation grid for the tracer density $\rho(R,|z| \mid p_\Phi, p_\text{DF})$ generated by the given qDF and according to \S\ref{sec:qDF} and Equation \ref{eq:tracerdensity}. For the creation of the mock data we use $N_x = 20$, $N_v = 40$ and $n_\sigma=5$. Next, we sample random positions $(R_i,z_i,\phi_i)$ uniformly within the entire observable volume. Then we apply a rejection Monte Carlo method to these positions using the pre-calculated $\rho_\text{DF}(R,|z| \mid p_{\Phi},p_\text{DF})$. To apply a non-uniform selection function, sf$(\vect{x}) \neq $ const. within the observed volume, we use the rejection method a second time. The resulting sample then follows $\vect{x}_i \longrightarrow p(\vect{x}) \propto \rho_\text{DF}(R,z \mid p_{\Phi},p_\text{DF}) \times \text{sf}(\vect{x})$.\\

In the second step we draw velocities according to the distribution function. The velocities are independent of the selection function within the observed volume. For each of the positions $(R_i,z_i)$ we sample velocities directly from the qDF$(R_i,z_i,\vect{v} \mid p_{\Phi},p_\text{DF})$ using a rejection method. To reduce the number of rejected velocities, we use a Gaussian in velocity space as an envelope function, from which we first randomly sample velocities and then apply the rejection method to shape the Gaussian velocity distribution towards the velocity distribution predicted by the qDF. We now have a mock data satisfying $(\vect{x}_i,\vect{v}_i) \longrightarrow p(\vect{x},\vect{v}) \propto \text{qDF}(\vect{x},\vect{v} \mid p_{\Phi},p_\text{DF}) \times \text{sf}(\vect{x})$.\\

%====================================================================

Figure \ref{fig:kks2WedgeEx} shows examples of mock data sets in configuration space $(\vect{x},\vect{v})$ and action space. The mock data from the qDF lead to the expected distributions in configuration space: More stars are found at smaller $R$ and $|z|$, and are distributed uniformly in $\phi$ according to our assumption of axisymmetry. The distribution in radial and vertical velocities, $v_R$ and $v_z$, is approximately Gaussian with the (total projected) velocity dispersion being $\sim\sigma_{R,0}$ and $\sim\sigma_{z,0}$ (see Table \ref{tbl:referenceMAPs}). The distribution of tangential velocities $v_T$ is skewed because of asymmetric drift. The distribution in action space illustrates the intuitive physical meaning of actions: The stars of the \texttt{cool} population have in general lower radial and vertical actions, as they are on more circular orbits. The different relative distributions of the radial and vertical actions $J_R$ and $J_z$ of the \texttt{hot} and \texttt{cool} population is due to them having different velocity anisotropy $\sigma_{R,0}/\sigma_{z,0}$. The different ranges of angular momentum $L_z$ in the two volumes reflect $L_z \sim R  v_\text{circ}$ and the different radial extent of both volumes. The volume above the plane contains stars with higher $J_z$, because stars with small $J_z$ cannot reach that far above the plane. Circular orbits with $J_R = 0$ and $J_z = 0$ can only be observed in the Galactic mid-plane. An orbit with $L_z$ much smaller or larger than $L_z(R_\odot)$ can only reach into a volume located around $R_\odot$, if it is more eccentric and has therefore larger $J_R$. This together with the effect of asymmetric drift can be seen in the asymmetric distribution of $J_R$ in the top central panel of Figure \ref{fig:kks2WedgeEx}.\\

If we want to add measurement errors to the mock data, we need to apply the following modifications to the above procedure. First, measurement errors are best described in heliocentric observables (see Section \ref{sec:coordinates}), we therefore assume and apply Gaussian errors to the \emph{true} phase-space coordinates $\tilde{\vect{x}} = (\text{RA},\text{DEC},(m-M)), \tilde{\vect{v}} = (\mu_\text{RA},\mu_\text{DEC},v_\text{los})$, where we have taken $(m-M)$ as a proxy for distance. Second, in the case of distance errors, stars can virtually scatter in and out of the observed volume. To account for this, we draw the \emph{true} positions from a volume that is larger than the actual observation volume, perturb the stars positions according to the distance errors and then reject all stars that lie now outside of the observed volume. This procedure mirrors the Poisson scatter around the detection threshold for stars whose distances are determined from the apparent brightness and the distance modulus. We then sample velocities (given the \emph{true} positions of the stars) as described above and perturb them according to the measurement errors as well.

