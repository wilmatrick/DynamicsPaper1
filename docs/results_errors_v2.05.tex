\subsection{Measurement Errors and their Effect on the Parameter Recovery} \label{sec:results_errors}

\Wilma{[TO DO: Comment from HW: This Section has three parts:}
\\\Wilma{-- convergence of the integral (ALREADY REMOVED)}
\\\Wilma{-- testing the approximation}
\\\Wilma{-- underestimating errors}
\\\Wilma{It seems to me that the basic Section:
What is the impact of the errors? Is missing. That should be the center piece, and the other three aspects should be quick summary notes, only 1-2 sentences long.]} \Wilma{[I'll try to address this with a plot mean(SE) vs. proper motion error - also for cold population (currently running on wolf).]}


In absence of distance uncertainties the error convolved likelihood given in Equation \ref{eq:errorconv} is unbiased.  When including distance (modulus) errors, Equation \ref{eq:errorconv} is just an approximation for the true likelihood. The systematic bias thus introduced in the parameter recovery gets larger with the size of the error. This is demonstrated in Figure \ref{test:isoSphFlexErrConv_bias_vs_SE}.  We find however that in case of $\delta(m-M) \lesssim 0.3 \text{ mag}$ (if also $\delta \mu \leq 2 \text{ mas yr}^{-1}$ and a maximum distance of $r_\text{max} = 3 \text{ kpc}$, see Test \ref{test:isoSphFlexErrConv_bias_vs_SE} in Table \ref{tbl:tests}) the potential parameters can still be recovered within 2 sigma \Wilma{[TO DO: Make sure this is what I claim in abstract and discussion.]}. This corresponds to a relative distance error of $\sim10\%$.

\Wilma{[TO DO: Introduce a test and plot that demonstrates how the SE depends on proper motion error. Then write this little section.]} Overall the standard errors on the recovered parameters are quite small (a few percent at most for 10,000 stars), which demonstrates that, if we perfectly knew the measurement errors, we still could get very precise constraints on the potential. The constraints also get tighter the smaller the proper motion error becomes. We found that for $\delta \mu = 1 \text{ mas yr}^{-1}$ the precision of the recovered parameters reduce by $\sim$ half compared to $\delta \mu = 5 \text{ mas yr}^{-1}$.

We found that in case we perfectly knew the measurement errors (and the distance error is negligible), the convolution of the model probability with the measurement uncertainties gives precise and accurate constraints on the model parameters - even if the error itself is quite large.

Figure \ref{fig:isoSphFlexErrSyst} now investigates the effect of a systematic \emph{underestimation} of the true proper motion uncertainties $\delta \mu$ by 10\% and 50\%. We find that this causes a bias in the parameter recovery that grows seemingly linear with $\delta \mu$. For an underestimation of only $10\%$ however, the bias is still $\lesssim 2$ sigma for 10,000 stars \Wilma{[TO DO: Check]} - even for $\delta \mu \sim 3~\text{mas yr}^{-1}$.

The size of the bias also depends on the kinematic temperature of the stellar population and the model parameter considered (see Figure \ref{fig:isoSphFlexErrSyst}). The qDF parameters are for example better recovered by hotter populations. This is, because the \emph{relative} difference between the true $\sigma_i(R)$ (with $i \in \{R,z\}$) and measured $\sigma_i(R)$ (which comes from the deconvolution with an underestimated velocity uncertainty) is smaller for hotter populations. 

%=============================================================

\begin{figure}
\plotone{figs/isoSphFlexErrConv_bias_vs_SE.eps}
\caption{Potential parameter recovery using the approximation for the model probability convolved with measurement uncertainties in Equation \ref{eq:errorconv}. We show  \pdf{} offset and relative width (i.e., standard error SE) for the potential parameters derived from mock data sets, which were created according to Test \ref{test:isoSphFlexErrConv_bias_vs_SE} in Table \ref{tbl:tests}). The data sets in the left panels have only uncertainties in line-of-sight velocity and proper motions, while the data sets in the right panels also have distance (modulus) uncertainties, as indicated in the legends in the first row. For data sets with proper motion error errors $\delta(m-M) \leq 3 \ \text{mas yr}^{-1}$ Equation \ref{eq:errorconv} was evaluated with $N_\text{error}=800$, for $\delta(m-M) > 3 \ \text{mas yr}^{-1}$ we used $N_\text{error}=1200$. In absence of distance uncertainties Equation \ref{eq:errorconv} gives unbiased results. For $\delta(m-M) \geq 3 \text{mas yr}^{-1}$ (which corresponds in this test to $\delta v_\text{max} \lesssim 43 \ \text{km s}^{-1}$, see Equation \ref{eq:vmax}) however biases of several sigma are introduced as Equation \ref{eq:errorconv} is only an approximation for the true likelihood in this case.}
\label{fig:isoSphFlexErrConv_bias_vs_SE}
\end{figure}

%=============================================================

\begin{figure}
\plotone{figs/Coming-Soon-Placeholder.eps}
\caption{\Wilma{[TO DO: This should be a figure that plots precision (SE) vs. proper motion error for a hot and a cool population (for no distance error). This is to demonstrate the effect of measurement errors in general. Currently running on cluster....]}}
\label{fig:???}
\end{figure}



%=============================================================

\begin{figure}
\plotone{figs/isoSphFlexErrSyst_offset_vs_error.eps}
\caption{Effect of a systematic underestimation of proper motion errors in the recovery of the model parameters. The true model parameters used to create the mock data are summarized as Test \ref{test:isoSphFlexErrSyst} in Table \ref{tbl:tests}, four of them are given on the $y$-axes and the true values are indicated as black dashed lines. The velocities of the mock data were perturbed according to Gaussian errors in the RA and DEC proper motions as indicated on the $x$-axis. The circles and triangles are the best fit parameters of several mock data sets assuming the proper motion uncertainty, with which the model probability was convolved, was underestimated in the analysis by 10\% or 50\%, respectively. The error bars correspond to 1 sigma confidence. The lines connect the mean of each two data realisations and are just to guide the eye. \Wilma{[TO DO: rename $h_{\sigma z}$ to $h_{\sigma,z}$, $\sigma_z$ to $\sigma_{z,0}$] [TO DO: Potential and/or population names in typewriter font] [TO DO: Iso-Pot in Title] [TO DO: Delta mu on x-axis]}}
\label{fig:isoSphFlexErrSyst}
\end{figure}

%=============================================================

\Wilma{[TO DO: Comment from Jo: Always use 'uncertainty' when describing how ou deal with the errors. 'Error' means the actual error (difference between observed and true).]}
