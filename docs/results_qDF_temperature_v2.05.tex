\subsection{The influence of the stellar population's kinematic temperature} \label{sec:results_temperature}

Overall, we found that it does not make a big and generic difference if we use hot or cool stellar populations in our modelling. Only to a certain extent the kinematic temperature plays a role for how precise and reliable model parameters can be recovered. 

While different populations constrain different parameters in different survey volumes with different precision, there is no easy rule of thumb, what combination would give the best results (see Figure \ref{fig:isoSph_CLT}).

There are two exceptions: 

First, the circular velocity at the Sun, $v_\text{circ}(R_\odot)$, is always best recovered with cooler populations (see Figures \ref{fig:isoSphFlexErrConv_SE_vs_error}, \ref{fig:isoSphFlexErrSyst}, \ref{fig:isoSphFlexMixCont}, \ref{fig:isoSphFlexMixDiff} and \ref{fig:MW14vsKKS2SphFlex}), because more stars are on near-circular orbits (see Figure \ref{fig:kks2WedgeEx}). As cooler populations probe the rotation curve better, which in turn probes the gravitational potential, the potential recovery using cool stellar populations is less sensitive to misjudgements of (spatial) selection functions (see Figures \ref{fig:isoSphFlexIncompR_violins} and \ref{fig:isoSphFlexIncompR_marginal_violins}).

Second, hotter populations seem to be less sensitive to misjudgements of proper motion measurement uncertainties (see Figure \ref{fig:isoSphFlexErrSyst}) and pollution with stars from a cooler population (see Figures \ref{fig:isoSphFlexMixCont} and \ref{fig:isoSphFlexMixDiff}), because of their higher intrinsic velocity dispersion (see Figure \ref{fig:kks2WedgeEx}).

In addition we found indications in Figure \ref{fig:MW14vsKKS2SphFlex}, that different regions within the Galaxy are probed best by populations of different kinematic temperature. The \texttt{hot} stellar population, with more stars reaching to high $|z|$ and a shorter tracer scale length, constrained force and density contours in the halo better---especially at smaller radii. The \texttt{cool} population, with more stars in the plane and longer tracer scale length, gave tighter force and density constraints in the outer regions of the halo and recovered the disk more reliably.