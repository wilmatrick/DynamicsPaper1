\clearpage
%\LongTables
\begin{landscape}
\begin{deluxetable}{llllll}
\tabletypesize{\scriptsize}
%\rotate
\tablecaption{Gravitational potentials of the reference galaxies used troughout this work and the respective ways to calculate actions in these potentials. All four potentials are axisymmetric. The potential parameters are fixed for the mock data creation at the values given in this table. In the subsequent analyses we aim to recover these potential parameters again. The parameters of "MW13-Pot" and "KKS-Pot" were found as direct fits to the "MW14-Pot". \label{tbl:referencepotentials}}
\tablewidth{0pt}
\tablehead{
\colhead{name} & \colhead{potential type} & \multicolumn{2}{c}{potential parameters $p_\Phi$} & \colhead{action calculation} & \colhead{reference for potential type}}
\startdata
"Iso-Pot" & isochrone potential   & circular velocity at the sun             & $v_\text{circ}$ = $230$ km s$^{-1}$           & \textbf{\emph{analytical and exact}} $J_r, J_\vartheta, L_z$;     & \citet{bin08} \\
          &					      & isochrone scale length                   & $b$ = $0.9$ kpc                               & use $J_r \rightarrow J_R, J_\vartheta \rightarrow J_z $  &               \\
          &                       &                                          &                                               & in eq. (???)                                             &               \\
\tableline
"KKS-Pot" & 2-component                  & circular velocity at the sun             & $v_\text{circ}$ = $230$ km s$^{-1}$           & \textbf{\emph{exact}} $J_R, J_z, L_z$       & \citet{bat94} \\
          & Kuzmin-Kutuzov-              & focal distance of coordinate system\tablenotemark{a}       & $\Delta = 0.3$              & using "St\"{a}ckel Fudge"                   &               \\                                                                
          & St\"{a}ckel potential        & axis ratio of the coordinate surfaces\tablenotemark{a} ... &                             & \citep{bin12}                               &               \\
          & \hspace{0.3cm} (disk + halo) & \hspace{0.3cm} ...of the disk component   & $\left(\frac{a}{c}\right)_\text{Disk}$ = 20  & and interpolation                           &               \\
          &                              & \hspace{0.3cm} ...of the halo component   & $\left(\frac{a}{c}\right)_\text{Halo}$ = 1.07& on action grid                              &               \\
          & (analytic potential)         & relative contribution of the disk mass    &                                              & \citep{bov15}                               &               \\
          &                              & \hspace{0.3cm} to the total mass          & $k = 0.28$                                   &                                             &               \\  
\tableline
"MW13-Pot" & MW-like potential with        & circular velocity at the sun             & $v_\text{circ}$ = $230$ km s$^{-1}$           & \textbf{\emph{approximate}} $J_R, J_z, L_z$ & \citet{bov13} \\          
           & Hernquist bulge,              & stellar disk scale length                & $R_d = 3$ kpc                                 & using "St\"{a}ckel Fudge"          &               \\
           & 2 exponential disks           & stellar disk scale height                & $z_h = 0.4$ kpc                               & \citep{bin12}                      &               \\
           & \hspace{0.3cm} (stars + gas), & relative halo contribution to $v_\text{circ}^2(R_\odot)$ & $f_h = 0.5$                   & and interpolation                  &               \\
           & spherical power-law halo      & "flatness" of rotation curve & $\frac{\diff \ln(v_\text{circ}(R_\odot))}{ \diff \ln(R)}$ = 0  & on action grid                &               \\
           & (interpolated potential)      &                                          &                                               & \citep{bov15}                      &               \\
\tableline
"MW14-Pot" & MW-like potential with        &  -                                       & -                                             & \textbf{\emph{approximate}} $J_R, J_z, L_z$ & \citet{bov15} \\
           & cut-off power-law bulge,       &                                          &                                               & (see "MW13-Pot")                   &               \\
           & Miyamoto-Nagai stellar disk,  &                                          &                                               &                                    &               \\
           & NFW halo                      &                                          &                                               &                                    &               \\
\enddata
\tablenotetext{a}{The coordinate system of each of the two St\"{a}ckel-potential components is $\frac{R^2}{\tau_{i,p}+\alpha_p} + \frac{z^2}{\tau_{i,p}+\gamma_p}=1$ with $p \in \{\text{Disk},\text{/Halo}\}$ and $\tau_{i,p} \in \{\lambda_p,\nu_p\}$. Both components have the same focal distance $\Delta = \sqrt{\gamma_p-\alpha_p}$, to make sure that the superposition of the two components itself is still a St\"{a}ckel potential. The axis ratio of the coordinate surfaces $\left(\frac{a}{c}\right)_p := \sqrt{\frac{\alpha_p}{\gamma_p}}$ describes the flattness of the corresponding St\"{a}ckel component.}
\end{deluxetable}

\begin{deluxetable}{lccccc}
\tabletypesize{\scriptsize}
%\rotate
\tablecaption{Reference distribution function parameters for the qDF in eq. (\ref{eq:df_general})-(\ref{eq:sigmazRg}). These qDFs describe the phase-space distribution of stellar \MAPs for which mock data is created and analysed throughout this work for testing purposes. The parameters of the "cooler" \& "colder"  ("hotter" \& "warmer") \MAPs were chosen such, that the they have the same $\sigma_R/\sigma_z$ ratio as the "hot" ("cool") \MAP. The "colder" and "warmer" \MAPs have a free parameter $X$ that governs how much colder/warmer they are then the reference "hot" and "cool" qDFs. Hotter populations have shorter tracer scale lengths \citep{bov12d} and the velocity dispersion scale lengths were fixed according to \citet{bov12c}. \label{tbl:referenceMAPs}}
\tablewidth{0pt}
\tablehead{
\colhead{name of \MAP} & \multicolumn{5}{c}{qDF parameters $p_\text{DF}$}\\
                       & \colhead{$h_R$ [kpc]} & \colhead{$\sigma_R$ [km s$^{-1}$]} & \colhead{$\sigma_z$ [km s$^{-1}$]} & \colhead{$h_{\sigma_R}$ [kpc]} & \colhead{$h_{\sigma_z}$ [kpc]}}
\startdata
"hot"    & 2   & 55 & 66 & 8 & 7\\
"cool"   & 3.5 & 42 & 32 & 8 & 7\\
\tableline
"cooler" & 2  +50\% & 55-50\% & 66-50\% & 8 & 7 \\
"hotter" & 3.5-50\% & 42+50\% & 32+50\% & 8 & 7\\
\tableline
"colder" & 2  +X\% & 55-X\% & 66-X\% & 8 & 7 \\
"warmer" & 3.5-X\% & 42+X\% & 32+X\% & 8 & 7\\
\enddata
\end{deluxetable}

\clearpage
\end{landscape}
