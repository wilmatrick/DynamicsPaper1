\subsection{Impact of misjudging the selection function of the data set} \label{sec:results_incompR}

The selection function (see Section \ref{sec:selectionfunction}) can be very complex and is therefore sometimes not perfectly known. Here we investigate how much this could affect the recovery of the potential. We do this by creating mock data in a spherical survey volume around the Sun (see Test \ref{test:isoSphFlexIncomp} in Table \ref{tbl:tests}) and a spatially varying completeness function
\begin{equation}
\text{completeness}(r) \equiv 1- \epsilon_r \frac{r}{r_\text{max}}, \label{eq:rad_incomp}
\end{equation}
which drops linearly with distance $r$ from the Sun. In the \RM{} analysis however, we assume constant completeness ($\epsilon_r=0$). The $\epsilon_r$ of the mock data quantifies therefore by how much we misjudge the SF. This captures the relevant case of stars being less likely to be observed (than assumed) the further away they are (e.g. due to unknown dust obscuration). 

Figure \ref{fig:isoSphFlexIncompR_violins} demonstrates that the potential recovery with \RM{} is very robust against somewhat wrong assumptions about the radial completeness of the data. The robustness for the \texttt{cool} stellar population is even more striking than for the \texttt{hot} population. The reason for this robustness could be, that much information about the potential comes from the rotation curve measurements in the plane, which is not affected by the incompleteness of the sample. We test this by not including tangential velocity measurements in the analysis of the data sets from Figure \ref{fig:isoSphFlexIncompR_violins} (by marginalizing the likelihood in Equation \ref{eq:prob} over $v_T$). Figure \ref{fig:isoSphFlexIncompR_marginal_violins} shows that in this case the potential is much less tightly constrained, even for 20,000 stars. For only small deviations of true and assumed completeness ($\lesssim 10\%$) we can however still incorporate the true potential in our fitting result (see Figure \ref{fig:isoSphFlexIncomp_marginal_violins}).

We found similarly robust results also for a misjudgement of spatial completeness functions varying with the distance from the plane, $|z|$.


%=============================

\Wilma{[TO DO: Look out for dead references]}