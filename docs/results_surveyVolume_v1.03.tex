\subsection{The Role of the Survey Volume Geometry} \label{sec:results_obsvolume}

Beyond the sample size, the survey volume {\it per se} must play a role; clearly, even a vast and perfect data set of stars within 100~pc of the Sun, has limited power to tell us about the potential at very different $R$. Intuitively, having dynamical tracers over a wide range in $R$ suggests to allow tighter constraints on the radial dependence of the potential. To this end, we devise two suites of mock data sets: 
\\The first one draws mock data from the same \pmodel (see Test \textcircled{4} in Table \ref{tbl:tests}), but from four different volume wedges (see \S\ref{sec:selectionfunction}), illustrated in the right upper panel of fig. \ref{fig:obsvolumetest}. To make the parameter inference comparison very differential, the mock data sets are equally large (20,000) in all cases, and are drawn from identical total survey volumes ($4.5~\text{kpc}^3$, achieved by adjusting the angular width of the edges). We perform the same test for three different potential models ("Iso-Pot", "MW14-Pot" and "KKS-Pot").
\\The results of this first mock suite are shown in Fig. \ref\ref{fig:wedFlexVol_bias_vs_SE} for all potential fit parameters. The mock suite was created to investigate the influence of \emph{position and shape} of the survey volume within the Galaxy, together with the \emph{choice of potential model}. 
\\The second suite of mock data sets was already introduced in \S\ref{sec:largedata} (see also Test \textcircled{3}), where mock data sets were drawn from five spherical volumes around the sun with different radius, for two "Iso-Pot"s and two different \MAPs.
\\The results of this second suite are shown in Fig. \ref{fig:centrallimittheorem} for one potential and one qDF parameter ($\ln(h_{\sigma,z})$) and investigates the effect of the \emph{size and extent} of the survey volume together with the \emph{choice of the \MAP and model parameters}.


The left panels of Fig.\ref{fig:wedFlexVol_bias_vs_SE} the illustrate the ability of \RM to constrain model parameters (in this case two $p_\Phi$ parameters). The two top right panels of Fig.\ref{fig:obsvolumetest} illustrate that the radial extent and the maximal height above the mid-plane matter. In the case shown, the standard error of the estimated parameters in twice as large for the volume with small $\Delta R$ and $\Delta |z|$; unsurprisingly, in the axisymmetic context the larger $\Delta\phi$ extent of that volume does not help to constrain the parameters. The panels in the bottom row explore whether the radial or vertical extent plays a dominant role: it appears that substantive radial and vertical extent are comparably important to constrain the parameters. 

This Figure also implies that for these cases volume offsets in the radial or vertical direction have at most modest impact. While we believe the argument for significant radial and vertical extent is generic, we have not done a full exploration of all combinations of \pmodel~ and volumina. Figure \ref{fig:centrallimittheorem}  amplifies the same point: it illustrates that at given sample size, drawing the data -- more sparsely -- from a larger volume provides better \pmodel~ constraints. 

%====================================================================

%FIGURE: Does shape and position of obs. volume matter?


\begin{figure}
\plotone{figs/wedFlexVol_bias_vs_SE.eps}
\caption{Bias vs. standard error in recovering the potential parameters for mock data stars drawn four different test observation volumes within the Galaxy (illustrated in the upper right panel) and three different potentials ("Iso-Pot", "MW13-Pot" and "KKS-Pot" from Table \ref{tbl:referencepotentials}, top to bottom row). Standard error and offset were determined as in fig. \ref{fig:centrallimittheorem}. Per volume and potential we analyse four different mock data realisations; all model parameters are shown as Test \textcircled{4} in Table \ref{tbl:tests}. The colour-coding represents the different wedge-shaped observation volumes. The angular extent of each wedge-shaped observation volume was adapted such that all have the volume of $4.5 \text{ kpc}^3$, even though their extent in $(R,z)$ is different.  Overall there is no clear trend, that an observation volume around the sun, above the disk or at smaller Galactocentric radii should give remarkably better constraints on the potential the other volumes. [TO DO: MW-Pot and KKS-Pot analyses suffer from too low accuracy in action calculation (with StaeckelGrid). Used the StaeckelGrid for BOTH mock data and analysis, but the mock data distribution would actually not look exactly like the desired qDF distribution, i.e. this plot basically is created with a messed up DF. Don't know how higher accuracy would change the plot. The orange Iso-Pot analysis suffers from too small integration range in vT. More coding required, before redoing this.]}
\label{fig:wedFlexVol_bias_vs_SE}
\end{figure}

 \paragraph{[TO DO] Stuff to explain about fig. \ref{fig:centrallimittheorem}:} Mention also that bigger volumes give most of the time better constraints and that there is no clear answer, if a hot or cooler population gives better constraints. Depends on parameter considered, selection function etc.
[TO DO] 'Larger is better' is also demonstrated in fig. \ref{fig:centrallimittheorem}
We demonstrate that for a given size of the observation volume the shape and position of the volume does not matter much as long as we have both large radial and/or vertical coverage. 
In an axisymmetric potential the coverage in angular direction does not matter, as long as there are enough stars in the observation volume.


%====================================================================
