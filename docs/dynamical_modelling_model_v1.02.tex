\subsection{Model}

\subsubsection{Actions}

[TO DO]

\subsubsection{Potential models}  \label{sec:potentials}

[TO DO] Mention different ways to calculate actions in different potentials.

%======================================================================================

\begin{deluxetable}{llllll}
\tabletypesize{\scriptsize}
\rotate
\tablecaption{Gravitational potentials of the reference galaxies used troughout this work and the respective ways to calculate actions in these potentials. All four potentials are axisymmetric. The potential parameters are fixed for the mock data creation. In the subsequent analyses we aim to recover these potential parameters again. All reference potentials assume the sun to be located at $(R_\odot,z_\odot)=(8~\text{kpc},0)$. \label{tbl:referencepotentials}}
\tablewidth{0pt}
\tablehead{
\colhead{name} & \colhead{potential type} & \multicolumn{2}{c}{potential parameters $p_\Phi$} & \colhead{action calculation} & \colhead{reference for potential type}}
\startdata
"Iso-Pot" & isochrone potential   & circular velocity at the sun             & $v_\text{circ}$ = $230$ km s$^{-1}$           & \textbf{\emph{analytical and exact}} $J_r, J_\vartheta, L_z$;     & \citet{bin08} \\
          &					      & isochrone scale length                   & $b$ = $0.9$ kpc                               & use $J_r \rightarrow J_R, J_\vartheta \rightarrow J_z $  &               \\
          &                       &                                          &                                               & in eq. (???)                                             &               \\
\tableline
"KKS-Pot" & 2-component           & circular velocity at the sun             & $v_\text{circ}$ = $230$ km s$^{-1}$           & \textbf{\emph{exact}} $J_R, J_z, L_z$       & \citet{bat94} \\
          & Kuzmin-Kutuzov-       & focal distance of coordinate system\tablenotemark{a}       & $\Delta = 0.3$              & using "St\"{a}ckel Fudge"                   &               \\                                                                
          & St\"{a}ckel potential & axis ratio of the coordinate surfaces\tablenotemark{a} ... &                             & \citep{bin12}                               &               \\
          &                       & \hspace{0.3cm} ...of the disk component   & $\left(\frac{a}{c}\right)_\text{Disk}$ = 20  & [in analysis: additional grid               &               \\
          &                       & \hspace{0.3cm} ...of the halo component   & $\left(\frac{a}{c}\right)_\text{Halo}$ = 1.07& interpolation \citep{bov15}]                &               \\
          &                       & relative contribution of the disk mass    &                                              &                                             &               \\
          &                       & \hspace{0.3cm} to the total mass          & $k = 0.28$                                   &                                             &               \\  
\tableline
"MW13-Pot" & MW-like potential with        & circular velocity at the sun             & $v_\text{circ}$ = $230$ km s$^{-1}$           & \textbf{\emph{approximate}} $J_R, J_z, L_z$ & \citet{bov13} \\          
           & Hernquist bulge,              & stellar disk scale length                & $R_d = 3$ kpc                                 & using "St\"{a}ckel Fudge"          &               \\
           & 2 exponential disks           & stellar disk scale height                & $z_h = 0.4$ kpc                               & \citep{bin12}                      &               \\
           & \hspace{0.3cm} (stars + gas), & relative halo contribution to $v_\text{circ}^2(R_\odot)$ & $f_h = 0.5$                   & [in analysis: additional grid      &               \\
           & spherical power-law halo      & "flatness" of rotation curve & $\frac{\diff \ln(v_\text{circ}(R_\odot))}{ \diff \ln(R)}$ = 0  & interpolation \citep{bov15}]  &               \\
\tableline
"MW14-Pot" & MW-like potential with        &  -                                       & -                                             & \textbf{\emph{approximate}} $J_R, J_z, L_z$                     & \citet{bov15} \\
           & cutoff power-law bulge,       &                                          &                                               & (see "MW13-Pot")                                   &               \\
           & Miyamoto-Nagai stellar disk,  &                                          &                                               &                                    &               \\
           & NFW halo                      &                                          &                                               &                                    &               \\
\enddata
\tablenotetext{a}{The coordinate system of each of the two St\"{a}ckel-potential components is $\frac{R^2}{\tau_{i,p}+\alpha_p} + \frac{z^2}{\tau_{i,p}+\gamma_p}=1$ with $p \in \{\text{Disk},\text{/Halo}\}$ and $\tau_{i,p} \in \{\lambda_p,\nu_p\}$. Both components have the same focal distance $\Delta = \sqrt{\gamma_p-\alpha_p}$, to make sure that the superposition of the two components itself is still a St\"{a}ckel potential. The axis ratio of the coordinate surfaces $\left(\frac{a}{c}\right)_p := \sqrt{\frac{\alpha_p}{\gamma_p}}$ describes the flattness of the corresponding St\"{a}ckel component.}
\end{deluxetable}

%=============================================

%FIGURE: reference potentials

\begin{figure}
\plotone{figs/reference_potentials.eps}
\caption{Density distribution of the four reference galaxy potentials in table \ref{tbl:referencepotentials}, for illustration purposes. These potentials are used throughout this work for mock data creation and potential recovery. [TO DO: Halo sichtbarer machen, evtl. mit isodensity contours]}
\label{fig:ref_pots}
\end{figure}

%=============================================

\subsubsection{Distribution function} \label{sec:qDF}

Motivated by the findings of Bovy et al. 2012??? and \citet{tin13} about the simple phase-space structure of \MAPs (see \S\ref{sec:intro}), and following \citet{bov13} and their successful application, we also assume that each \MAP follows a single qDF of the form given by \citet{bin11}.  This qDF  is a function of the actions $\vect{J}=(J_R,J_z,L_z)$ and has the form
\begin{eqnarray}
\text{qDF}(\vect{J} \mid p_\text{DF}) &=& f_{\sigma_R}\left(J_R,L_z \mid p_\text{DF}\right) \times f_{\sigma_z}\left(J_z,L_z \mid p_\text{DF}\right)\label{eq:df_general}\\
\text{with } f_{\sigma_R}\left(J_R,L_z \mid p_\text{DF}\right) &=& n \times \frac{\Omega}{\pi\sigma_R^2(R_g) \kappa}\left[1+\tanh\left(L_z/L_0\right) \right]\exp\left(-\frac{\kappa J_R}{\sigma_R^2(R_g)} \right) \\
f_{\sigma_z}\left(J_z,L_z \mid p_\text{DF} \right) &=& \frac{\nu}{2 \pi \sigma_z^2(R_g)} \exp\left( -\frac{\nu J_z}{\sigma_z^2(R_g)} \right) \\
\end{eqnarray}
Here $R_g \equiv R_g(L_z)$ and $\Omega\equiv \Omega(L_z)$ are the (guidig-center) radius and the circular frequency of the circular orbit with angular momentum $L_z$ in a given potential. $\kappa\equiv \kappa(L_z)$ and $\nu\equiv \nu(L_z)$ are the radial/epicycle ($\kappa$) and vertical ($\nu$) frequencies with which the star would oscillate around the circular orbit in $R$- and $z$-direction when slightly perturbed \citep{bin08}. The term $\left[1+\tanh\left(L_z/L_0\right) \right]$ suppresses counter-rotation for orbits in the disk with $L \gg L_0$ which we set to a random small value ($L_0 = 10 \times R_\odot/8 \times v_\text{circ}(R_\odot)/220$).
\\For this qDF to be able to incorporate the findings by Bovy et al. 2012??? about the phase-space structure of \MAPs summarized in \S\ref{sec:intro}, we set the functions $n$,  $\sigma_R$ and $\sigma_z$, which indirectly set the stellar number density and radial and vertical velocity dispersion profiles,
\begin{eqnarray}
n(R_g) &\propto& \exp\left(-\frac{R_g}{h_R} \right)\\
\sigma_R(R_g) &=& \sigma_{R,0} \times \exp\left(- \frac{R_g-R_\odot}{h_{\sigma_R}} \right)\\
\sigma_z(R_g) &=& \sigma_{z,0} \times \exp\left(- \frac{R_g-R_\odot}{h_{\sigma_z}} \right)\label{eq:sigmazRg}.
\end{eqnarray}
The qDF for each \MAP has therefore a set of five free parameters $p_\text{DF}$: the density scale length of the tracers $h_R$, the radial and vertical velocity dispersion at the solar position $R_\odot$, $\sigma_R,0$ and $\sigma_z,0$, and the scale lengths $h_{\sigma_R}$ and $h_{\sigma_z}$, that describe the radial decrease of the velocity dispersion. The \MAPs we use for illustration through out this work are summarized in table ???.

%======================================================================================

\begin{deluxetable}{lccccc}
\tabletypesize{\scriptsize}
%\rotate
\tablecaption{Reference distribution function parameters for the qDF in eq. (\ref{eq:df_general})-(\ref{eq:sigmazRg}). These qDFs describe the phase-space distribution of stellar \MAPs for which mock data is created and analysed throughout this work for testing purposes. The parameters of the "cooler" ("hotter") \MAPs were chosen such, that the they have the same $\sigma_R/\sigma_z$ ratio as the "hot" ("cool") \MAP. Hotter populations have shorter tracer scale lengths \citep{bov12d} and the velocity dispersion scale lengths were fixed according to \citet{bov12c}. \label{tbl:referenceMAPs}}
\tablewidth{0pt}
\tablehead{
\colhead{name of \MAP} & \multicolumn{5}{c}{qDF parameters $p_\text{DF}$}\\
                       & \colhead{$h_R$ [kpc]} & \colhead{$\sigma_R$ [km s$^{-1}$]} & \colhead{$\sigma_z$ [km s$^{-1}$]} & \colhead{$h_{\sigma_R}$ [kpc]} & \colhead{$h_{\sigma_z}$ [kpc]}}
\startdata
"hot"    & 2   & 55 & 66 & 8 & 7\\
"cool"   & 3.5 & 42 & 32 & 8 & 7\\
\tableline
"cooler" & 2  +50\% & 55-50\% & 66-50\% & 8 & 7 \\
"hotter" & 3.5-50\% & 42+50\% & 32+50\% & 8 & 7\\
\enddata
\end{deluxetable}

%======================================================================================



[TO DO] [To Do here: Also mention how the density is calculated.]

\subsubsection{Selection function: observed volume and completeness}

[TO DO]
