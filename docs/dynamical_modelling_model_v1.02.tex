\subsection{Model}

\subsubsection{Actions}

[TO DO]

\subsubsection{Potential models}  \label{sec:potentials}

[TO DO] Mention different ways to calculate actions in different potentials.

%======================================================================================

\begin{deluxetable}{llllll}
\tabletypesize{\scriptsize}
\rotate
\tablecaption{Caption [TO DO].\label{tbl:referencepotentials}}
\tablewidth{0pt}
\tablehead{
\colhead{name} & \colhead{potential type} & \multicolumn{2}{c}{free parameters $p_\Phi$} & \colhead{action calculation} & \colhead{reference}}
\startdata
"Iso-Pot" & isochrone potential   & circular velocity at the sun             & $v_\text{circ}$ = $230$ km s$^{-1}$           & \textbf{\emph{analytical and exact}} $J_r, J_\vartheta, L_z$;     & \citet{bin08} \\
          &					      & isochrone scale length                   & $b$ = $0.9$ kpc                               & use $J_r \rightarrow J_R, J_\vartheta \rightarrow J_z $  &               \\
          &                       &                                          &                                               & in eq. (???)                                             &               \\
\tableline
"KKS-Pot" & 2-component           & circular velocity at the sun             & $v_\text{circ}$ = $230$ km s$^{-1}$           & \textbf{\emph{exact}} $J_R, J_z, L_z$                & \citet{bat94} \\
          & Kuzmin-Kutuzov-       & focal distance of coordinate system\tablenotemark{a}       & $\Delta = 0.3$              & using "St\"{a}ckel Fudge"                   &               \\                                                                
          & St\"{a}ckel potential & axis ratio of the coordinate surfaces\tablenotemark{a} ... &                             & \citep{bin12}                               &               \\
          &                       & \hspace{0.3cm} ...of the disk component   & $\left(\frac{a}{c}\right)_\text{Disk}$ = 20  & and interpolation on action grid            &               \\
          &                       & \hspace{0.3cm} ...of the halo component   & $\left(\frac{a}{c}\right)_\text{Halo}$ = 1.07& \citep{bov15}                               &               \\
          &                       & relative contribution of the disk mass    &                                              &                                             &               \\
          &                       & \hspace{0.3cm} to the total mass          & $k = 0.28$                                   &                                             &               \\  
\tableline
"MW13-Pot" & MW-like potential with        & circular velocity at the sun             & $v_\text{circ}$ = $230$ km s$^{-1}$           & \textbf{\emph{approximate}} $J_R, J_z, L_z$ & \citet{bov13} \\          
           & fixed Hernquist bulge,        & stellar disk scale length                & $R_d = 3$ kpc                                 & using "St\"{a}ckel Fudge"          &               \\
           & 2 exponential disks           & stellar disk scale height                & $z_h = 0.4$ kpc                               & \citep{bin12}                      &               \\
           & \hspace{0.3cm} (stars + gas), & relative halo contribution to $v_\text{circ}^2(R_\odot)$ & $f_h = 0.5$                   & and interpolation on action grid   &               \\
           & spherical power-law halo   & "flatness" of rotation curve & $\frac{\diff \ln(v_\text{circ}(R_\odot))}{ \diff \ln(R)}$ = 0  & \citep{bov15}                 &               \\
\tableline
"MW14-Pot" & MW-like potential with        &  -                                       & -                                             & see "MW13-Pot"                     & \citet{bov15} \\
           & cutoff power-law bulge,       &                                          &                                               &                                    &               \\
           & Miyamoto-Nagai stellar disk,  &                                          &                                               &                                    &               \\
           & NFW halo                      &                                          &                                               &                                    &               \\
\enddata
\tablenotetext{a}{The coordinate system of each of the two St\"{a}ckel-potential components is $\frac{R^2}{\tau_{i,p}+\alpha_p} + \frac{z^2}{\tau_{i,p}+\gamma_p}=1$ with $p \in \{\text{Disk},\text{/Halo}\}$ and $\tau_{i,p} \in \{\lambda_p,\nu_p\}$. Both components have the same focal distance $\Delta = \sqrt{\gamma_p-\alpha_p}$, to make sure that the superposition of the two components itself is still a St\"{a}ckel potential. The axis ratio of the coordinate surfaces $\left(\frac{a}{c}\right)_p := \sqrt{\frac{\alpha_p}{\gamma_p}}$ describes the flattness of the corresponding St\"{a}ckel component.}
\end{deluxetable}

%=============================================

%FIGURE: reference potentials

\begin{figure}
\plotone{figs/reference_potentials.eps}
\caption{[TO DO]}
\label{fig:ref_pots}
\end{figure}

%=============================================

\subsubsection{Distribution function} \label{sec:qDF}

The structure of the MW disk is still under debate: While many still support the thin-thick disk dichotomy in the MW disk (references ???), \citet{bov12b} found indications that the MW disk might actually be a super-position of many stellar sub-popluations with a continuous spectrum of scale heights, scale lengths, metallicity and [$\alpha$/Fe] abundances (dubbed mono-abundance populations (\MAPs)). Further investigation lead to the findings that \MAPs in the MW disk have a simple spatial structure that follows an exponential in both radial and vertical direction \citep{bov12d}. The corresponding velocity dispersion profile of the \MAPs also decreases exponentially with radius and is nearly independent of height above the plane, i.e. quasi-isothermal \citep{bov12c}. The radial decrease in vertical velocity dispersion has, according to \citet{bov12c}, a long scale length of $h_{\sigma,z} \sim 7$ kpc for all \MAPs. Older \MAPs, which are characterized by lower metallicities and [$\alpha$/Fe] abundances, have in general shorter density scale lengths, larger scale heights and velocity dispersion \citep{bov12d}. \citet{tin13} and \citet{bov13} finally proposed that these findings could be employed for dynamical modelling techniques using action-based distribution functions. An action-based distribution function, that is flexible enough to describe the spectrum of simple phase-space distributions of different \MAPs, is the quasi-isothermal distribution function (qDF) by \citet{bin11}, as demonstrated by \citet{tin13}.  The qDF by \citet{bin11} is a function of the actions $\vect{J}=(J_R,J_z,L_z)$ and has the form
\begin{eqnarray*}
\text{qDF}(\vect{J} \mid p_\text{DF}) &=& f_{\sigma_R}\left(J_R,L_z \mid p_\text{DF}\right) \times f_{\sigma_z}\left(J_z,L_z \mid p_\text{DF}\right)\\
\text{with } f_{\sigma_R}\left(J_R,L_z \mid p_\text{DF}\right) &=& n \times \frac{\Omega}{\pi\sigma_R^2(R_g) \kappa}\left[1+\tanh\left(L_z/L_0\right) \right]\exp\left(-\frac{\kappa J_R}{\sigma_R^2(R_g)} \right) \\
f_{\sigma_z}\left(J_z,L_z \mid p_\text{DF} \right) &=& \frac{\nu}{2 \pi \sigma_z^2(R_g)} \exp\left( -\frac{\nu J_z}{\sigma_z^2(R_g)} \right) \\
\end{eqnarray*}
Here $R_g \equiv R_g(L_z)$ and $\Omega\equiv \Omega(L_z)$ are the (guidig-center) radius and the circular frequency of the circular orbit with angular momentum $L_z$ in a given potential. $\kappa\equiv \kappa(L_z)$ and $\nu\equiv \nu(L_z)$ are the radial/epicycle ($\kappa$) and vertical ($\nu$) frequencies with which the star would oscillate around the circular orbit in $R$- and $z$-direction when slightly perturbed \citep{bin08}. The term $\left[1+\tanh\left(L_z/L_0\right) \right]$ suppresses counter-rotation for orbits in the disk with $L \gg L_0$ which we set to a random small value ($L_0 = 10 \times R_\odot/8 \times v_\text{circ}(R_\odot)/220$).
\\For this qDF to be able to incorporate the findings by Bovy et al. 2012 about the phase-space structure of \MAPs summarized above, we set the functions $n$,  $\sigma_R$ and $\sigma_z$, which indirectly set the stellar number density and radial and vertical velocity dispersion profiles,
\begin{eqnarray*}
n(R_g) &\propto& \exp\left(-\frac{R_g}{h_R} \right)\\
\sigma_R(R_g) &=& \sigma_{R,0} \times \exp\left(- \frac{R_g-R_\odot}{h_{\sigma_R}} \right)\\
\sigma_z(R_g) &=& \sigma_{z,0} \times \exp\left(- \frac{R_g-R_\odot}{h_{\sigma_z}} \right).
\end{eqnarray*}
The qDF for each \MAP has therefore a set of five free parameters $p_\text{DF}$: the density scale length of the tracers $h_R$, the radial and vertical velocity dispersion at the solar position $R_\odot$, $\sigma_R,0$ and $\sigma_z,0$, and the scale lengths $h_{\sigma_R}$ and $h_{\sigma_z}$, that describe the radial decrease of the velocity dispersion. The \MAPs we use for illustration through out this work are summarized in table ???.

%======================================================================================

\begin{deluxetable}{lccccc}
\tabletypesize{\scriptsize}
%\rotate
\tablecaption{Caption [TO DO]. The parameters of the "cooler" ("hotter") \MAPs were chosen such, that the they have the same $\sigma_R/\sigma_z$ ratio as the "hot" ("cool") \MAP. Hotter populations have shorter tracer scale lengths \citep{bov12d} and the velocity dispersion scale lengths were fixed according to \citet{bov12c}. \label{tbl:referenceMAPs}}
\tablewidth{0pt}
\tablehead{
\colhead{name of \MAP} & \multicolumn{5}{c}{free parameters $p_\text{DF}$}\\
                       & \colhead{$h_R$ [kpc]} & \colhead{$\sigma_R$ [km s$^{-1}$]} & \colhead{$\sigma_z$ [km s$^{-1}$]} & \colhead{$h_{\sigma_R}$ [kpc]} & \colhead{$h_{\sigma_z}$ [kpc]}}
\startdata
"hot"    & 2   & 55 & 66 & 8 & 7\\
"cool"   & 3.5 & 42 & 32 & 8 & 7\\
\tableline
"cooler" & 2  +50\% & 55-50\% & 66-50\% & 8 & 7 \\
"hotter" & 3.5-50\% & 42+50\% & 32+50\% & 8 & 7\\
\enddata
\end{deluxetable}

%======================================================================================



[TO DO] [To Do here: Also mention how the density is calculated.]

\subsubsection{Selection function: observed volume and completeness}

[TO DO]
