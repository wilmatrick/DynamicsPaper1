\section{Introduction} \label{sec:intro}

%Everything in a nutshell
Stellar dynamical modelling can be employed to infer the Milky Way's gravitational potential from the positions and motions of individual stars (\citealt{2008gady.book.....B,2011Prama..77...39B,2013A&ARv..21...61R}). Observational information on the 6D phase-space coordinates of stars is currently growing at a rapid pace, and will be taken to a whole new level in number and precision by the upcoming data from the Gaia mission \citep{2001A&A...369..339P}. Yet, rigorous and practical modelling tools that turn position-velocity data of individual stars into constraints both on the gravitational potential and on the distribution function (DF) of stellar orbits, are scarce \citep{2013A&ARv..21...61R} \Wilma{[TO DO: more references] [TO DO: References that explain that the modelling is scarce, or previous modelling approaches???] [TO DO: Hans-Walter suggested a Sanders \& Binney reference, but I'm still not sure to what kind of paper: modelling approach or review of scarce modelling tools...]}\\


%Quote Hans-Walter: There are two options (find someone else who said it; some recent "review");or say: There are only very few modelling tools [REF: ..list them..], and all have important limitations "Galactic potential is fundamental for dark matter / baryonic structure." In such a case, I would just list a few recent papers that have illustrate that importance.. http://adsabs.harvard.edu/abs/2014JPhG...41f3101R  Read Rix & Bovy 13 McMillan 2012 ... But is is also OK to leave very general statements unreferenced...


%Papers In which I am looking for references:
%http://arxiv.org/pdf/1104.2839v1.pdf
%http://arxiv.org/pdf/1301.3168v1.pdf, Rix & Bovy 2013, continue on page 13
%http://arxiv.org/pdf/1108.1749v2.pdf
%http://arxiv.org/pdf/1308.6386v1.pdf Review Zjelko

%Why potential + DF are important
The Galactic gravitational potential is fundamental for understanding the Milky Way's dark matter and baryonic structure \citep{2013A&ARv..21...61R,2012EPJWC..1910002M,2013PhR...531....1S,2014JPhG...41f3101R} and the stellar-population dependent orbit distribution function is a basic constraint on the Galaxy's formation history \citep{2013NewAR..57...29B,2013A&ARv..21...61R,2015MNRAS.449.3479S} \HW{[TO DO: more references]}.\\

%Action-based DF modelling in general
There is a variety of practical approaches to dynamical modelling of discrete collisionless tracers, such as the stars in the Milky Way (e.g. Jeans modelling: \citet{1989MNRAS.239..605K}, \citet{2012ApJ...756...89B}, \citet{2012MNRAS.425.1445G}, \citet{2013ApJ...772..108Z}, \citet{2015MNRAS.452..956B}; action-based DF modelling: \citet{2013ApJ...779..115B}, \citet{2014MNRAS.445.3133P}, \citet{2015MNRAS.449.3479S}; torus modelling:  \citet{2012MNRAS.419.2251M,2013MNRAS.433.1411M}; Made-to-measure modelling: \citet{sye96}, \citet{2007MNRAS.376...71D} or \citet{2014MNRAS.443.2112H}. Most of them -- explicitly or implicitly -- describe the stellar distribution through a distribution function. \\

%The roots of our approach
Actions are good ways to describe orbits, because they are canonical variables with their corresponding angles, have immediate physical meaning, and obey adiabatic invariance \citep{2008gady.book.....B,2008MNRAS.390..429M,2010MNRAS.401.2318B,2011MNRAS.413.1889B,2011Prama..77...39B}. Recently, \citet{2012MNRAS.426.1328B} and \citet{2013ApJ...779..115B} \HW{[TO DO: are these the correct references???]} proposed to combine parametrized axisymmetric potentials with DF's that are simple analytic functions of the three orbital actions to model discrete data. \citet{2010MNRAS.401.2318B} and \citet{2011MNRAS.413.1889B} had proposed a set of simple action-based (quasi-isothermal) distribution functions (qDF). \citet{2013MNRAS.434..652T} and \citet{2013ApJ...779..115B} showed that these qDF's may be good descriptions of the Galactic disk, when one only considers so-called mono-abundance populations (\MAP{}), i.e. sub-sets of stars with similar [Fe/H] and [$\alpha$/Fe] \citep{bov12b,bov12c,2012ApJ...753..148B}. \\

%The first version of the code + first results
\citet{2013ApJ...779..115B} implemented a rigorous modelling approach that put action-based DF modelling of the Galactic disk in an axisymmetric potential in practice. Given an assumed potential and an assumed DF, they directly calculated the likelihood of the observed ($\vec{x},\vec{v}$) for each sub-set of \MAP{} among SEGUE G-dwarf stars \citep{2009AJ....137.4377Y}. This modelling also accounted for the complex, but known selection function of the kinematic tracers.  For each \MAP{}, the modelling resulted in a constraint of its DF, and an independent constraint on the gravitational potential, which members of all \MAPs{} feel the same way. \\
Taken as an ensemble, the individual \MAP{} models constrained the disk surface mass density over a wide range of radii ($\sim 4-9$ kpc), and proved a powerful constraint on the disk mass scale length and on the disk-to-dark-matter ratio at the Solar radius. \\

%Drawbacks of the first code version in the era of large surveys
Yet, these recent models still leave us poorly prepared with the wealth and quality of the existing and upcoming data sets. This is because \citet{2013ApJ...779..115B} made a number of quite severe and idealizing assumptions about the potential, the DF and the knowledge of observational effects (such as the selection function). All these idealizations are likely to translate into systematic error on the inferred potential or DF, well above the formal error bars of the upcoming data sets. \\

%Focus of this work: Not just follow-up of BR13, but presentation and investigation of much improved machinery
In this work we present \RM{} (``\textsc{R}ecovery of the \textsc{O}rbit \textsc{A}ction \textsc{D}istribution of \textsc{M}ono-\textsc{A}bundance \textsc{P}opulations and \textsc{P}otential \textsc{IN}ference for our \textsc{G}alaxy'') - an improved and refined version of the original dynamical modelling machinery by \citet{2013ApJ...779..115B}, making extensive use of the \emph{galpy} Python package \citep{2015ApJS..216...29B} and the \emph{St\"{a}ckel Fudge} for fast action calculations by \cite{2012MNRAS.426.1324B}. \RM{} is robust and well-tested and explicitly developed to exploit and deal with the large data sets of the future. \RM{} explores and relaxes some of the restraining assumptions that \citet{2013ApJ...779..115B} made and is more flexible and more adept in dealing with large data sets. In this paper we set out to explore the robustness of \RM{} against the breakdowns of some of the most important assumptions of DF-based dynamical modelling. Our goal is to examine which aspects of the data, the model and the machinery itself limit our recovery of the true gravitational potential.\\

%Large Data + Machinery
In the light of the imminent Gaia data, we analyze how well \RM{} behaves in the limit of large data. For a huge number of stars three aspects become important, that may be hidden behind Poisson noise for smaller data sets: (i) We have to make sure that \RM{} is an unbiased estimator (Section \ref{sec:largedata}). (ii) Numerical inaccuracies in the actual modelling machinery must not be an important source of systematics (Section \ref{sec:likelihood}). (iii) As parameter estimates become much more precise (Section \ref{sec:largedata}, we need more flexibility in the potential and DF model. The modelling machinery therefore has to effective in finding the best fit parameters for a large set of free model parameters. The improvements made in \RM{} as compared to the machinery used in \citet{2013ApJ...779..115B} are presented in Section \ref{sec:fitting}.\\

%Data 
We also explore how different aspects of the observational experiment design impact the parameter recovery. (i) In an era where we can choose data from different MW surveys, it might be worth to explore the importance of the survey volume geometry, size and shape, and if different regions within the MW might be especially diagnostic to constrain the potential (Section \ref{sec:results_obsvolume}). (ii) What if our knowledge of the sample selection function is imperfect, and potentially biased (Section \ref{sec:results_incompR})? (iii) How to best account for individual measurement errors in the modelling (Section \ref{sec:results_errors})? \\

%Model
One of the strongest assumptions is to restrict the dynamical modelling to a certain family of parametrized models. We investigate how well we can we hope to recover the true potential, when our potential and DF models do not encompass the true potential and DF. First, we examine in Section \ref{sec:results_mixedDFs} what would happen if the stars within \MAPs{} do intrinsically not follow a single qDF as assumed by \citet{2013MNRAS.434..652T} and \citet{2013ApJ...779..115B}. Second, we test in Section \ref{sec:results_potential} how well the modelling works, if our assumed potential family deviaties from the true potential.\\

The strongest assumption that goes into this kind of dynamical modelling might be the idealization of the Galaxy to be axi-symmetric and being in steady state. We do not investigate this within the scope of this paper but strongly suggest a systematic investigation of this for future work.\\

For all of the above aspects we show some plausible and illustrative examples on the basis of investigating mock data. The mock data is generated from galaxy models presented in Sections \ref{sec:coordinates}-\ref{sec:selectionfunction} following the procedure in Section \ref{sec:mockdata}, analysed according to the description of the \RM{} machinery in Sections \ref{sec:likelihood}-\ref{sec:fitting} and the results are presented in Section \ref{sec:results} and discussed in Section \ref{sec:discussionsummary}.

\Wilma{[TO DO: Comment from Hans-Walter: Make sure, any topic/issue appears only once]}
\\\Wilma{[TO DO: Is now one quarter shorter than before. But maybe shorten it even more...]}
\\\Wilma{[TO DO: Comment from Hans-Walter: Make clear "new in this paper", "general background", "exactly as in BR13"]}
