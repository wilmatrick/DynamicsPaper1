\section{Introduction} \label{sec:intro}

%Everything in a nutshell
Dynamical modelling can be employed to infer the Milky Way's (MW) gravitational potential from stellar motions (\citealt{2008gady.book.....B,2011Prama..77...39B,2013A&ARv..21...61R}). Observational information on the 6D phase-space coordinates of stars is currently growing at a rapid pace, and will be taken to a whole new level in number and precision by the upcoming data from the Gaia mission \citep{2001A&A...369..339P}. Yet, rigorous and practical modelling tools that turn position-velocity data of individual stars into constraints both on the gravitational potential and on the distribution function (DF) of stellar orbits are scarce \citep{2013A&ARv..21...61R}.

%Why potential + DF are important
The Galactic gravitational potential is fundamental for understanding the MW's dark matter and baryonic structure \citep{2013A&ARv..21...61R,2012EPJWC..1910002M,2013PhR...531....1S,2014JPhG...41f3101R} and the stellar-population dependent orbit DF is a basic constraint on the Galaxy's formation history \citep{2013NewAR..57...29B,2013A&ARv..21...61R,2015MNRAS.449.3479S}.

%Action-based DF modelling in general
There is a variety of practical approaches to dynamical modelling of discrete collisionless tracers, such as the stars in the MW, e.g., Jeans modelling (\citealt{1989MNRAS.239..605K}; \citealt{2012ApJ...756...89B}; \citealt{2012MNRAS.425.1445G}; \citealt{2013ApJ...772..108Z}; \citealt{2015MNRAS.452..956B}), action-based DF modelling (\citealt{2013ApJ...779..115B}; \citealt{2014MNRAS.445.3133P}; \citealt{2015MNRAS.449.3479S}), torus modelling (\citealt{2008MNRAS.390..429M}; \citealt{2012MNRAS.419.2251M}; \citealt{2013MNRAS.433.1411M}), made-to-measure modelling (\citealt{1996MNRAS.282..223S}; \citealt{2007MNRAS.376...71D}; or \citealt{2014MNRAS.443.2112H}). Most of them--explicitly or implicitly--describe the stellar distribution through a DF. 

%The roots of our approach
Recently, \citet{2012MNRAS.426.1328B} and \citet{2013ApJ...779..115B} proposed to constrain the MW's gravitational potential by combining parametrized axisymmetric potential models with DFs that are simple analytic functions of the three orbital actions (\citealt{2008gady.book.....B}, \S 3.5 \& \S 4.6; \citealt{2011Prama..77...39B}) to model discrete data.



%The first version of the code + first results
\citet{2013ApJ...779..115B} (BR13 hereafter) put this in practice by implementing a rigorous modelling approach for so-called mono-abundance populations (\MAPs{}), i.e, sub-sets of stars with similar $[\mathrm{Fe}/\mathrm{H}]$ and $[\alpha/\mathrm{Fe}]$ within the Galactic disk, which seem to allow simple DFs \citep{2012ApJ...751..131B,2012ApJ...755..115B,2012ApJ...753..148B}. Given an assumed (axisymmetric) model for the Galactic potential and action-based DF \citep{2010MNRAS.401.2318B,2011MNRAS.413.1889B,2013MNRAS.434..652T} they calculated the likelihood of the observed ($\vec{x},\vec{v}$) for each \MAP{} among SEGUE G-dwarf stars \citep{2009AJ....137.4377Y}. They also accounted for the complex, but known selection function of the kinematic tracers \citep{2012ApJ...753..148B}. For each \MAP{} the modelling resulted in an independent estimate on the same gravitational potential. Taken as an ensemble, they constrained the disk surface mass density over a wide range of radii ($\sim 4-9~\text{kpc}$), and proved to be a powerful constraint on the disk mass scale length and on the disk-to-dark-matter ratio at the Solar radius. 

%Drawbacks of the first code version in the era of large surveys
BR13 made however a number of quite severe and idealizing assumptions about potential, DF and the knowledge of observational effects. These idealizations are likely to translate into systematic errors on the inferred potential, well above the formal error bars of the upcoming surveys with their wealth and quality of data.

%Focus of this work: Not just follow-up of BR13, but presentation and investigation of much improved machinery
In this work we present \RM{} (``\textsc{R}ecovery of the \textsc{O}rbit \textsc{A}ction \textsc{D}istribution of \textsc{M}ono-\textsc{A}bundance \textsc{P}opulations and \textsc{P}otential \textsc{IN}ference for our \textsc{G}alaxy'')---an improved, refined, flexible, robust and well-tested version of the original dynamical modelling machinery by BR13, explicitly developed to deal with large data sets. Our goal is to explore which of the assumptions BR13 made and which other aspects of data, model and machinery limit \RM{}'s recovery of the true gravitational potential.

%Large Data + Machinery
We investigate the following aspects of the \RM{} machinery that become especially important for a large number of stars: (i) Numerical inaccuracies must not be an important source of systematics (Section \ref{sec:likelihood}). (ii) As parameter estimates become much more precise, we need more flexibility in the potential and DF model and effective strategies to find the best fit parameters. The improvements made in \RM{} as compared to the machinery used in BR13 are presented in Section \ref{sec:fitting}. (iii) We have to make sure that \RM{} is an unbiased estimator (Section \ref{sec:largedata}).

%Data 
We also explore how different aspects of the observational experiment design impact the parameter recovery: (i) It might be worth to explore the importance of the survey volume geometry, size, shape and position within the MW to constrain the potential (Section \ref{sec:results_obsvolume}). (ii) What if our knowledge of the sample selection function is imperfect, and potentially biased (Section \ref{sec:results_incompR})? (iii) How to best account for individual and possibly misjudged measurement uncertainties (Section \ref{sec:results_errors})? (iv) Given several stellar sub-populations of different kinematic temperature---what is the best choice (Section \ref{sec:results_temperature})? 

%Model
One of the strongest assumptions is to restrict the dynamical modelling to a certain family of parametrized models. We investigate how well we can hope to recover the true potential, when our models do not encompass the true DF (Section \ref{sec:results_mixedDFs}) and potential (Section \ref{sec:results_potential}).

The most severe idealization that goes into this kind of dynamical modelling might be that of the Galaxy being axisymmetric and in steady state. We do not investigate this within the scope of this paper, but strongly suggest a systematic investigation of this for future work.

For all of the above aspects we show some plausible and illustrative examples on the basis of investigating mock data. The mock data is generated from galaxy models presented in Sections \ref{sec:coordinates}-\ref{sec:selectionfunction} following the procedure in Section \ref{sec:mockdata}, analysed according to the description of the \RM{} machinery in Sections \ref{sec:likelihood}-\ref{sec:fitting}. The results on the investigated modelling aspects are presented in Section \ref{sec:results} and summarized and discussed in Section \ref{sec:discussionsummary}.