\section{Introduction} \label{sec:intro}

[TO DO]

\paragraph{Collection of thoughts for the introduction:} \textit{(Text is not yet perfect or concise, but should serve as a starting point to setup a basic structure for the introduction. The text will then have to be shortened, redundant formulations have to be removed, phrasing has to be improved and everything has to be supported with appropriate references.)}
\begin{itemize}
\item \textsc{RoadMapping} stands for "Recovery of the Orbit/Action Distribution of Mono-Abundance Populations and Potential INference for our Galaxy".
\item \textbf{Our modelling method in a nutshell:} We fit simultaneously a model for the Galaxy's gravitational potential and an orbit distribution function (df) to stellar phase-space data. To turn a star's position and velocity into a full orbit, we need the gravitational potential in which the star moves. We assume that we know a family of orbit distribution functions that are close enough to the real distribution of orbits. In this case the stellar orbits calculated within a proposed potential will only follow such a df, if this potential model is close enough to the true potential.
\\Or in other words: We need the potential to calculate orbits. At the same time, if we \textit{know} the true orbits, we can deduce the true potential from them. To find the true orbits, we make use of the predictive power of an orbit distribution function.

\item \textbf{Introducing orbits and actions:} There are different ways to describe stellar orbits. The most obvious is to give the stars position and velocity vector at each point in time, by evaluating the potential forces that act on the star in each time step. Most orbits in realistic galaxy potentials  are however not closed, so we would have to integrate the orbit forever. Another, much more convenient way to describe orbits, are so called integrals of motion. These integrals are functions of the star's time-dependent position and velocity, but are themselves constants in time, i.e. conserved quantities. The most obvious integral in static potentials is the energy of the orbit. Symmetries in potentials frequently allow more than one integral: In spherical potentials all three components of the angular momentum are conserved. In many axisymmetric potentials there is, in addition to the energy $E$ and vertical component of the angular momentum $L_z$, a third non-classical integral of motion $I_3$, which has however no easy physical meaning.  (Binney \& Tremaine, Galactic Dynamics)\\
Because any function of integrals is an integral of motion itself, it is possible to construct integrals that have both very convenient properties and intuitive physical meanings. One such a set are the so-called actions. In axisymmetric potentials they are frequently called the radial action $J_R$, the vertical action $J_z$ and the $\phi$-action, which is simply the vertical component of the angular momentum, $L_z$. The radial action and vertical action quantify the amount of oscillation in radial and vertical direction that the orbit exhibits.  Actions are constructed in such a way, that they are not only integrals, but also correspond to the momenta in a set of canonical coordinates. The canonical conjugate positions of the actions are the so-called angles, which have the convenient properties, that they increase strictly linearly in time while the star moves along the orbit. They are periodic in $2\pi$ and the frequencies by which they change are functions of the actions. In the action-angle coordinate system, the only thing we need to fully describe an orbit in an axisymmetric potential are therefore just three fixed numbers, the actions. 

\item \textbf{Using actions for distribution functions:} Actions are therefore the natural coordinates of orbits and each point in action space corresponds to one specific orbit in a given potential. It is often used in dynamical modelling, e.g. in the Schwarzschild superposition method (source???), to reconstruct a galaxy by superimposing different orbits and populating them with stars. In this way these kind of methods construct orbit distribution functions for galaxies, which are at the same time distribution functions in action space. Because angles increase linearly in time, when a star moves along its orbit, stars are uniformly distributed in angle space. Therefore a orbit distribution function in terms of actions and a uniform distribution of stars in angle-space can be directly mapped to a distribution of stars in canonical configuration phase-space, measurable stellar positions and velocities. While a stellar distribution in configuration space is six-dimensional, the distribution in action-angle space is effectively three-dimensional, because of the uniformity in angles. (Rewrite, too verbose...)

\item \textbf{Motivation to use Binney's qDF in the modelling:} Astronomers consider galaxies frequently as a superposition of several components. The stellar component itself is often separated into bulge, halo and several disk sub-populations. Distribution functions in terms of actions have the advantage, that a full distribution function for the whole galaxy can be constructed by a superposition of action-based DFs for each component. [TO DO: This was what Payel Das told me, but I forgot why this is the case and I also didn't find any reference for this. ???] Assuming a DF with a simple form for each galaxy component can give, in superposition, very realistic looking, flexible and successful models for the distribution of stars in galaxies \citep{bov13,san15,pif14} (other references???). Any dynamical modelling approach still depends crucially the assumption one makes about the structure of the galaxy and on the choices for the DFs:
\\The structure of the MW disk is still under debate. While many still support the thin-thick disk dichotomy in the MW disk (references ???), \citet{bov12b} found indications that the MW disk might actually be a super-position of many stellar sub-popluations with a continuous spectrum of scale heights, scale lengths, metallicity and [$\alpha$/Fe] abundances (dubbed mono-abundance populations (\MAPs)). Further investigation lead to the findings that \MAPs in the MW disk have a simple spatial structure that follows an exponential in both radial and vertical direction \citep{bov12d}. The corresponding velocity dispersion profile of the \MAPs also decreases exponentially with radius and is nearly independent of height above the plane, i.e. quasi-isothermal \citep{bov12c}. The radial decrease in vertical velocity dispersion has, according to \citet{bov12c}, a long scale length of $h_{\sigma,z} \sim 7$ kpc for all \MAPs. Older \MAPs, which are characterized by lower metallicities and [$\alpha$/Fe] abundances, have in general shorter density scale lengths, larger scale heights and velocity dispersion \citep{bov12d}. \citet{tin13} and \citet{bov13} finally proposed that these findings could be employed for dynamical modelling techniques using action-based distribution functions. An action-based distribution function, that is flexible enough to describe the spectrum of simple phase-space distributions of different \MAPs, is the quasi-isothermal distribution function (qDF) by \citet{bin11}, as demonstrated by \citet{tin13}.

\item \textbf{Some caveats of DF assumptions as compared to others:} \cite{san15} and ??? develop extended distribution functions (EDFs), that extend action-based DFs to also describe the distribution of the star's metallicities. While a full chemo-dynamical modelling,  including metallicity as well as $\alpha$- and other chemical abundances, is ultimately the right way to go, the form of the EDFs still depends on a lot of additional assumptions. By looking at fig. 6 in \cite{bov13} (other references???) we doubt that a final version of an EDF will have a simple form in action-metallicity space. Motivated by the findings by Bovy et al. 2012, we therefore resign to the simpler approach outlined in \cite{bov13} and here, were metallicity and $\alpha$-abundances are implicitly taken into account by describing each MAP separately by one qDF. This procedure could have two caveats: 
\\First, the binning of the stars according to their abundances could lead to pollution of one MAP, by either choosing the bin sizes too large, or too small compared to the stars' inherent abundance errors. 
\\Second, while \citet{tin13} makes us confident that the qDF is indeed a good functional form to describe each MAP, it could very well be, that the stars' true distribution is close to but not exactly of the family of assumed qDFs. 
\\\textbf{Some comments by HWR regarding Sanders \& Binneys take on our modelling, should be also included: [TO DO]} 
\begin{itemize}
\item Overall, there is no doubt that making a simultaneous model for the "chemo-orbital-potential" distribution has some advantages over the "orbital-potential" distribution at a given abundance. The main advantage for pursueing MAP modelling at least as a first/intermediate step is: a) it separates out complexity (i.e. it's much easier to "see" what goes right or wrong), b) it provides true cross-checking redundancy w.r.t. to the potential estimates [TO DO: I don't understand the latter]
\item "First, choosing bin sizes always requires a compromise between losing the information contained in the position of each datum within its bin and increasing Poisson noise by making the bins small." --> That is true for any binning. But with the realistic samples sizes, bins within which the abundances vary "little" have a sensible number of stars (for SEGUE)
\item It is true that the MAP approach does not exploit that the abundance space distribution is "smooth"; however, the data show that there is no "simple and large-scale" pattern that lends itself to a simple functional form.
\item " Third, we require errors in the ([Fe/H], [alpha/Fe]) space that are much smaller than the bin sizes, otherwise we are neglecting the possibility of contamination on each bin by neighbouring bins." --> I don't think the argument is valid; it wouldn't make sense to make the bins SMALLER than the errors, because then you would reduce the samples size and increase the shot-noise, WITHOUT making the approximation to the model DF better. You could make the bin size larger (therefore the error smaller than the bin-size), but would pay the prize of a poorer approximation. So, I actually would think that making the bin size of order of the abundance error is a sensible choice, if that leaves you with "enough" objects in the bin.
\item " Additionally, a continuous parametrization allows for a rigorous treatment of the error distributions in ([Fe/H], [alpha/Fe]) and how these errors correlate with the kinematic errors. Hence we believe that it is best to work with an EDF provided we are confident that we have a sufficiently flexible and well-tailored functional form." -->I would agree with that statement but a) it's not easy to get a simple form for that, see sigma\_z(FeH,aFe) Figures in B12; and the redundance argument from above applies...
\end{itemize}

\item \textbf{Why should we care about actions in realistic galaxies?} In reality galaxies have rarely perfectly static and axisymmetric potentials, which drastically reduces the number of conserved quantites along orbits. In static non-axisymmetric  potentials there can still be two integrals of motion, angular momentum however is no longer conserved. The Milky Way's disk might have an overall axisymmetric appearance, but is perturbed by spiral arms. The strongest deviation from axisymmetry in the Galaxy is the bar, which also causes the Galactic potential to vary slowly in time. The stirrs up the stars of the disk and the potential and causes radial migration of the orbits (Reference???), orbits change and with them the actions. One could wonder if, under such non-axisymmetric, non-static potential conditions, the assumption and treatment of globally conserved actions in the Milky Way is still a sensible approach. First of all, actions are the natural way to treat orbits and they can be locally defined, even if they might not be globally conserved. As long as we care about orbits, we should care about actions. An orbit carries information about the star's past, about where the star was born and which tidal processes might have carried it away from its inital orbit. Together with the chemistry of the stars, which determined by their place of birth, their current orbits are valuable diagnostics for the evolution and structure of the Milky Way. Secondly, gravitational processes do only in the most extreme cases completely change the actions. In a slowly changing potential, where orbits adapt adiabatically to those changes, actions are conserved (Binney \& Tremaine, Galactic Dynamics). And even during bar-induced radial migration at least the vertical actions are conserved and will continue to carry some amount of information about the stars' inital orbit distribution.\\

[TO DO] (Maybe cite Potzen 2015, who showed that analysing aspherical systems in spherical actions can still be a powerful tool, when used with care...)

\item \textbf{Why should we care about an axisymmetric "best fit" model for the Milky Way disk?} One of the key assumptions of our modelling technique is the assumed axisymmetry of the Milky Way's gravitational potential, especially its disk. As we discussed already in the previous paragraph, this assumption is indeed only an approximation to the real disk, which has a much richer structure and more complicated potential, with spiral arms and ring-like structures (like the Monocerros ring), with a warp and a flare in the outer disk (references????). Also the Milky Way's halo has substructure, a multitude of streams (references???) and shell-like overdensities (reference???). The ultimate goal will be to find and identify substructures observationally and describe theoretically the structure and evolution of potential perturbations. Our method and efforts  to extract information about the axisymmetric Milky Way potential from disk stars aims to create a reliable and well-constraint basis for these endevours: The best possible axisymmetric approximation to the Milky Way's potential could serve as a realistic equilibrium model from which a description of non-axisymmetric tidal perturbations can be theoretically established by perturbation theory. It will also help a great deal to identify sub-structures, e.g. to find and orbitally connect tidal streams, which in return will then give better constraints on the deviations from axisymmetry. Many modelling and techniques, both purely gravitational, but also chemo-dynamical, can greatly profit from a good axisymmetric model for the galaxy: While we are still far away from knowing the MW's potential all over the place, an axisymmetric model will be the best reference to turn phase-space coordiantes into whole orbits. And orbits are the diagnostics that carry information from everywhere in the galaxy into the solar neighbourhood, where we can hope to exploit them. (Some overlap with section before. How to better structure these two sections and assign the arguments more clearly to "axisymmetric disk" or "actions"?) 

\item \textbf{Previous results with this modelling technique:} \citet{bov13} ... [TO DO]
\begin{itemize}
\item disk scale length $R_d = 2.15 \pm 0.14 \text{ kpc }$ \citep{bov13}
\item disk is maximal \citep{bov13}
\item slope of dark matter halo $\alpha < 1.53$ \citep{bov13}
\end{itemize}

\item \textbf{What do we already know about the axisymmetric MW disk (from other references)?} [TO DO]
\begin{itemize}
\item rotation curve is well-known (reference???)
\end{itemize}

\item \textbf{What is there left to learn about the axisymmetric MW disk?} (as Jo asked at the Santa Barbara conference... [TO DO]
\begin{itemize}
\item separation of different MW component is still unclear: individual density profiles, contributions to total pot
\item thin/thick disk vs. continuum of exponential disks
\item dark matter at smaller radii
\item slope \& shape of dark matter halo (current state of knowledge?)
\end{itemize} 

\item \textbf{Other modelling approaches:}
\begin{itemize}
\item \citet{pif14} used a slightly different DF-based modelling approach to constrain the MW's vertical density profile near the sun. They fitted a superposition of "quasi-isothermal" DFs for thick and thin disk, and a DF for the halo to \textasciitilde200,000 giant stars from the RAVE survey (RAdial Velocity Experiment, \citet{ste06}). They didn't use any chemical information of the stars. To account for different populations within the thin disk, they weighted the corresponding DF's with an assumed star-formation rate instead. To circumvent the use of RAVE's non-trivial spatial selection function, they separated stars into spatial bins in $(R,z)$ and fitted the velocity distribution predicted by their DF and potential model at the mean $(R,z)$ of each bin to the observed velocities only. They're result for their radial profile of the vertical force within $|z|=1.1$ kpc and $R>6.6$ kpc agrees well with the previous results from our method by \citet{bov13}. By not using chemical information and hiding the spatial distribution of stars by binning to circumvent a complicated selection function, \citet{pif14} is however rejecting a lot of valuable information in the data set. ([TO DO: Look at other useful references in this paper: Bienayme et al. 2014, Zhang et al. 2013, Binney et al. 2014a, Binney 2012b, McMillan \& Binney 2013])
\end{itemize}



\item \textbf{Motivating this method characterization in anticipation of GAIA:} [TO DO]


\item \textbf{Ideas how to structure this introduction:}
\begin{itemize}
\item Part I: Basic task is fitting potential and DF at the same time. This is a great, useful and successful way to constrain the galactic gravitational potential. Was already done in \citet{bov13}.
\item Part II: While \citet{bov13} were successful in their application, they made many approximations / assumptions / idealisations, which were not tested for their valicity and might actually not hold up well. We want to investigate this. We cannot test everythong, but we show some plausible and illustrative examples (often using a spherical isochrone potential for convenience). (Also mention what \citet{san15} say about this modelling approach.) 
\end{itemize}

\end{itemize}
