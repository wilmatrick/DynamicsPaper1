\begin{abstract}
Starting point for abstract: my old poster abstract. [TO DO] We aim to recover the Milky Way's gravitational potential using action-based dynamical modeling (cf. Bovy \& Rix 2013, Binney \& McMillan 2011, Binney 2012). This technique works by modeling the observed positions and velocities of disk stars with an equilibrium, three-integral quasi-isothermal distribution function. In preparation for the application to stellar phase-space data from Gaia, we create and analyze a large suite of mock data sets and we develop qualitative "rules of thumb" for which characteristics and limitations of data, model and code affect constraints on the potential most. We investigate sample size and measurement errors of the data set, size and shape of the observed volume, numerical accuracy of the code and action calculation, and deviations of the data from the assumed family of axisymmetric model potentials and distribution functions. This will answer the question: What kind of data gives the best and most reliable constraints on the Galaxy's potential?
\end{abstract}