\subsection{Effect of measurement errors on recovery of potential?} \label{sec:results_errors}

[TO DO]

\paragraph{Collection tests and plots (tests are still running on the cluster)}

\begin{itemize}
\item \emph{Plot 1:} number of MC samples needed for the error convolution vs. maximum velocity error inside the observed volume, such that a given accuracy in potential and qDF parameters is reached. Similar to what I had on the poster. However, we still haven't tested, if this plot depends on: hotness of stars and or umber of stars.

\item \emph{Plot 2:} 2 columns of panels (one row for each parameter), bias vs. standard error. First column: only proper motion and vlos errors $\longrightarrow$ shows that our error convolution works and should be bias free, plus, when knowing the errors perfectly we can get a perfect deconvolution and tight constraints. Second column: proper motion, vlos and distance modulus errors $\longrightarrow$ shows that for too large proper motion and distance errors our approximation for the error convolution does not work anymore.

\item \emph{Plot 3:} 2 or 4 columns of panels (one row for each parameter), bias vs. standard error. First column: only proper motion errors. Errors used in analysis are systematically 10\% smaller than the actual measurement errors. Second column: only proper motion errors. Errors used in analysis are randomly smaller than actual measurement errors (mean: 10\%, Gaussian distributed). Third \& fourth column: The same thing but for a mock data set with different temperature. $\longrightarrow$ This plot should demonstrate how large the measurement errors can be (assuming we get them wrong by 10\% anyway), until we are not able to recover the true parameter anymore.

\end{itemize}
