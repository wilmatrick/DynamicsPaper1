\subsection{Mock Data}

One goal of this work is to test how the loss of information in the process of measuring stellar phase-space coordinates can affect the outcome of the modelling. To investigate this, we assume first that our measured stars do indeed come from our assumed families of potentials and distribution functions and draw mock data from a given true distribution. In further steps we can manipulate and modify these mock data sets to mimick observational effects.\\
The distribution function is given in terms of actions and angles. The obvious procedure would be to draw $\vect{J}_i$ from qDF$(\vect{J}_i \mid p_\text{DF})$ and $\vect{\theta}_i$ between 0 and $2\pi$, transforming this to $(\text{x}_i,\text{v}_i)$ in the given potential and rejecting all stars that are outside the observed volume. The transformation $(\vect{J}_i,\vect{\theta}_i) \longrightarrow (\vect{x}_i,\vect{v}_i)$ is however difficult to perform and computationally much more expensive than the transformation $(\vect{x}_i,\vect{v}_i) \longrightarrow (\vect{J}_i,\vect{\theta}_i)$. The observed volume is also much smaller than the whole galaxy and the fraction of rejected stars would be enormous. We propose a fast and simple two-step method for drawing mock data from an action distribution function in a given observed volume.
[TO DO]

\subsubsection{Preparation: Tracer density} \label{sec:density}

[TO DO: This section should be shifted to the distribution function section. ???]

One crucial point in creating the mock data, as well as in the analysis, is to calculate the spatial tracer density $\rho_\text{DF}(\vect{x} \mid p_{\Phi},p_\text{DF})$ for a given distribution function with parameters $p_\text{DF}$ in a potential with parameters $p_{\Phi}$. We do this by integrating the axisymmetric distribution function over the velocity at $N_\text{spatial} \times N_\text{spatial}$ regular grid points in the $(R,z)$ plane, using a Gauss-Legendre quadrature of order $N_\text{velocity}$ in each of the three velocity components. The velocity distribution according to the qDF at a given $(R_i,z_i)$ looks approximately Gaussian for $v_R$ and $v_z$. The $v_T$ distribution peaks somewhere around $v_\text{circ}(R_\odot)$ and $v_T > 0$ (cf. \S ?????). We approximate the integration over the velocity therefore as
\begin{eqnarray}
\rho_\text{DF}(R,|z| \mid p_{\Phi},p_\text{DF}) &=& \int_{-\infty}^{\infty} \text{qDF}(J[R,z,\vect{v} \mid p_{\Phi}] \mid p_\text{DF}) \Diff3\vect{v}  \label{eq:tracerdensity_general}\\
&\approx& \int_{-N_\text{sigma} \sigma_R(R \mid p_\text{DF})}^{N_\text{sigma} \sigma_R(R \mid p_\text{DF})} \int_{-N_\text{sigma}\sigma_z(R \mid p_\text{DF})}^{N_\text{sigma} \sigma_z(R \mid p_\text{DF})} \int_{0}^{1.5 v_\text{circ}(R_\odot)}  \nonumber\\
& & \hspace{1cm} \text{qDF}(J[R,z,\vect{v} \mid p_{\Phi}] \mid p_\text{DF}) \diff v_T \diff v_z \diff v_R, \label{eq:tracerdensity}
\end{eqnarray}
where $\sigma_R(R \mid p_\text{DF})$ and $\sigma_z(R \mid p_\text{DF})$ are the star's radial and vertical velocity dispersion according to the qDF and given by eq. (???) and (???) and $N_\text{sigma}$ has to be large enough. The size of the $N_\text{spatial} \times N_\text{spatial}$ grid in $(R,z)$ is chosen cover the extent of the observed volume for $z>0$. We interpolate then over this grid to be able to evaluate the density at each $(R,z)$ within the observed volume. The total number of grid points and therefore actions that need to be calculated are $N_\text{spatial}^2 \cdot N_\text{velocity}^3$. Fig. ??? shows the importance of choosing $N_\text{spatial}$, $N_\text{velocity}$ and $N_\text{spatial}$ sufficiently large in order to get the density with an acceptable numerical accuracy. For the creation of the mock data we use $N_\text{spatial} = 20$, $N_\text{velocity} = 40$ and $N_\text{sigma}=5$.

\subsubsection{Step 1: Drawing positions from the selection function}

To get $\vect{x}_i$ for our mock data stars, we first sample random positions $(R_i,z_i,\phi_i)$ uniformly from the observed volume. Making use of the geometric shapes of our simple observed volumes (spheres, cylinders, wedges...) we can do this efficiently with inverse transform Monte Carlo sampling. In a second step we then apply a rejection Monte Carlo method to these positions using the pre-calculated, interpolated density grid $\rho_\text{DF}(R,|z| \mid p_{\Phi},p_\text{DF})$. In an optional third step, if we want to apply a non-uniform selection function, sf$(\vect{x}) \neq $ const. within the observed volume, we use the rejection method a second time on the set of positions we got from the last step. This procedure can be repeated until the required number of positions are reached. The sample then follows the required distribution
\begin{equation*}
\vect{x}_i \longrightarrow p(\vect{x}) \propto \rho_\text{DF}(R,z \mid p_{\Phi},p_\text{DF}) \times \text{sf}(\vect{x}).
\end{equation*}

\subsubsection{Step 2: Drawing velocities according to the distribution function}

The velocities are independent of the selection function and observed volume. For each of the positions $(R_i,z_i)$ found in step 1 we now sample velocities directly from the qDF$(R_i,z_i,\vect{v} \mid p_{Phi},p_\text{DF})$ using a rejection method. To reduce the number of rejected velocities, we use a Gaussian in velocity space as an envelope function, from which we first randomly sample velocities and then apply the rejection method to shape the Gaussian velocity distribution towards the velocity distribution predicted by the qDF. The envelope Gaussian peaks at $(v_R=0,v_T=max(\text{qDF}(R_i,z_i,0,vT,0 \mid p_{Phi},p_\text{DF})),v_z=0)$ and has a standard deviation of $(2\sigma_R(R_i),2\sigma_R(R_i),2\sigma_z(R_i))$. We then pick one of the sampled velocities $\vect{v}_i$ at and for each $(R_i,z_i)$ and arrive at our desired mock data sample
\begin{equation*}
(\vect{x}_i,\vect{v}_i) \longrightarrow p(\vect{x},\vect{v}) \propto \text{qDF}(\vect{x},\vect{v} \mid p_{\Phi},p_\text{DF}) \times \text{sf}(\vect{x}).
\end{equation*} 
[TO DO: mention fig. \ref{fig:mockdatadistr}. ???]

%====================================================================

%FIGURE: distribution of mock data in action and configuration space

\begin{figure}[H]
\plotone{figs/kks2WedgeEx_mockdata.eps}
\caption{Distribution of mock data in action space (2D iso-density contours in the two central and the lower left panel) and configuration space (1D histograms in right panels), depending shape and position of observation volume (cf. lower left panel) and 'hottness' of the stellar population. The mock data was created in a 2-component KK-Staeckel-potential (cf. ???) with parameters $p_\Phi = \{v_\text{circ}, \Delta, (a/c)_\text{disk},(a/c)_{halo},k)\} = \{230\text{ km s$^{-1}$},0.3,20.,1.07, 0.28\}$ [TO DO: Does $\Delta$ have units????] (which is an approximate fit to the MilkyWay2014 potential in Galpy). We use two stellar populations, a 'hot' one with $p_{DF,hot} = \{ h_R, \sigma_R, \sigma_z,h_{\sigma_R},h_{\sigma_z}\} =\{2 \text{ kpc}, 55 \text{ km s$^{-1}$}, 66 \text{ km s$^{-1}$}, 8 \text{ kpc}, 7 \text{ kpc }\} $ (red lines) and a 'cool' population with $p_{DF,cool} = \{ h_R, \sigma_R, \sigma_z,h_{\sigma_R},h_{\sigma_z}\} =\{3.5 \text{ kpc}, 42 \text{ km s$^{-1}$}, 32 \text{ km s$^{-1}$}, 8 \text{ kpc}, 7 \text{ kpc }\} $ (blue lines). In the upper left panel we demonstrate the shape of the two different observation volumes within which we were creating each a 'hot' and a 'cool' mock data set: a large volume centered on the Galactic plane (solid lines in all plots) and a smaller one above the plane (dashed lines in all plots). Both volumes have an angular extent of $\Delta\phi \pm 20^\circ$. Each of the four mock data sets compared in this plot has 20,000 stars in it. The stars of the 'cool' population have in general lower radial and vertical actions, i.e. are on more circular obrits. The different ranges of $L_z$'s in the two volumes reflect $L_z \sim R  v_\text{circ}$ and the different radial extent of both volumes.The volume above the plane contains no stars with $J_z = 0$ and more with $J_z$: The higher a volume is located above the plane, the larger $J_z$ has to be for the star's orbit to cross this volume. Circular orbits with $J_R = 0$ and $J_z = 0$ can obviously only be observed in the Galactic mid-plane. The smaller an orbit's $L_z$, the smaller also its mean orbital radius. For this orbit to be able to reach into a volume located at larger Galacto-centric radius, it needs to be more eccentric and therefore have a larger $J_z$. This anti-correlation between $L_z$ and $J_R$ can be seen in the top central panel. Orbits with both large $J_R$ and large $J_z$ would be very energetic and are therefore less likely to be observed. [TO DO: How many percent do the contours enclose????]} 
\label{fig:mockdatadistr}
\end{figure}
%
%\paragraph{[TO DO] Stuff to explain about fig. \ref{fig:mockdatadistr}:} Shift description of action space from caption to here. Mention, that the distribution according to the qdf in configuration space is indeed how expected: More stars at lower R and lower $|z|$, uniform in $\phi$. Using the velocity dispersion (with $\sigma_0$ being indeed the velocity dispersion at the sun) as integration ranges over $v_R$ and $v_z$ is fine. Maybe overplot $\pm5\sigma_0$ range? For $v_T$ distribution: Explain asymmetric drift. Would $v_T$ for a potential with perfectly flat rotation curve peak at 230km/s? This potential has only an approximately flat rotation curve.\\Cool and hot have different anisotropy --> difference between cool and hot in $J_R$ and $J_z$ is therefore different. \\Asymmetric drift at a given radius: 1) density profile: more stars inside as compared to outside. stars from inside come from smaller radius and are currently at their apocenter --> slower as circular velocity. --> more likely to be observed as stars from outside at their pericenter. 2) dispersion profile: stars on larger radii are on more circular orbits, because lower velocity dispersion. --> there are even less pericenter stars from the outside in the sample. more stars on non-circular orbits in the inside --> more likely to reach out.

%===========================================================================================================================================================================================


\subsubsection{Step 3: Introducing measurement errors}

[TO DO]

\paragraph{[TO DO] Possible plots:} *Diagram*: schematic flow chart of how to sample mock data (could be helpful for people, who want to sample mock data in action space and didn't know how to start, like me)

