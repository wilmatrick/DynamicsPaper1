\section{Discussion and Summary} \label{sec:discussionsummary}

We present \RM, an improved implementation of the dynamical modelling machinery by \citet{bov13}, to recover the potential and orbit distribution function of stellar \MAPs within the Galactic disk. In this work we investigated the capabilities, strengths and weaknesses of \RM by testing its robustness against the breakdown of some of its assumptions - for well defined, isolated test cases using mock data. Overall the method works very well and reliable, also if there are small deviations of the model assumptions from the real world galaxy.

\subsection{Improved Implementation for Large Data Sets}

\RM applies a full likelihood analysis and is statistically well-behaved. It allows for a straightforward implementation of different potential model families and a flexible number of free fit parameters in potential and qDF. It also accounts for selection effects by using full 3D selection functions (given some symmetries). \RM is an asymptotically normal, un-biased estimator and the precision of parameter recovery increases by $1/\sqrt{N}$ with the number of stars.\\

Large data sets in the age of Gaia require more, and more accurate, likelihood evaluations for more flexible models. To be able to deal with these increased computational demands and explore larger parameter spaces, we sped up the code by combining a nested grid approach with MCMC and by faster action calculation using the St\"{a}ckel \citep{bin12} interpolation grid by \citet{bov15}. Especially accurately determining the likelihood normalisation will be of crucial importance for large data sets. The nested-grid approach automatizes the search for the optimal normalisation integration ranges ("fiducial qDF") and start position for the MCMC walkers, which helps the MCMC to converge fast and to reduce biases due to insufficient accuracy. However, application of \RM to millions of stars simultaneously with acceptable accuracy will still be a task for supercomputers and calls for even more improvements and speed-up in the fitting machinery.

\subsection{Characteristics of the Data}


\paragraph{Choice of observation volume.} We found that the \emph{position} of the survey volume maters little, in the sense that there are no regions in the Galaxy that contain intrinsically stars on manifestly more diagnostic orbits than others. Closer to the disk and at smaller Galactocentric radii it is only the increased number of stars that will lead to tighter constraints. Concerning the \emph{shape} of the survey volume, a large radial \emph{and} vertical coverage is best. In the axisymmetric case phi coverage doesn't matter. Making a volume cut for stars, that lie around $R_\odot$ but at larger $phi$, could therefore improve the results, if their measurements are very uncertain.
\\\MAPs of different scale length and temperature probe different regions of the Galaxy \citep{bov13}. But there is no easy rule of thumb for which survey volume and stellar population which potential and DF parameter is constrained best.

\paragraph{Selection function misjudgment.} Surprisingly \RM seems to be very robust against misjudgments in the selection function of the data. The reason for this robustness could be, that missing stars in the data set do not affect the connection between a star's velocity and position, which is given by the potential. A lot of information about the potential profile is stored in the rotation curve - but even when not including measurements of tangential velocities in the analysis, small misjudgments of the incompleteness do not affect the potential recovery.
\\That we reproduce the qDF equally well, could be due to the symmetry of our assumed incompleteness profiles around the sun. We investigated a decrease in knowledge of the data completeness in distance from the sun and Galactic plane. Our test with the radial incompleteness profile could be understood as a decreasing detection rate due to the lower apparent brightness of stars at larger distances. The test with the planar incompleteness profile could mimic a misjudgment of the dust obscuration within the Galactic plane. Both effects would show the same symmetries as tested in this work. 
\\This result is encouraging for future studies, but nevertheless surprising as it was previous believed that knowing the (spatial) selection function very precisely is of large importance for dynamical modelling \citep{rix13}.

\paragraph{Measurement errors.} Properly convolving the likelihood with measurement errors is computational very expensive. By ignoring positional errors and only including distance errors as part of the velocity error, we can drastically reduce the computational costs. [TO DO - No definite result after this disclaimer:] For stars within 3 kpc from the sun this approximation works well for distance errors of $\sim 10\%$ or smaller. The number of MC samples needed for the error convolution using MC integration scales by $N_\text{error} \propto (\sigma_{v,\text{max}})^2$ with the maximum velocity error at the edge of the sample. If we misjudge the size of the true measurement errors, we only can reproduce the true model parameters, as long as the velocity errors are smaller than the intrinsic velocity dispersion of the data set.

\subsection{Characteristics of the Model}

\paragraph{Misjudgment of the DF.} Our modelling crucially depends on the assumption, that the Galactic disk consists of many \MAPs, each following a qDF. In Example 1 in \S ??? we investigated how well we can recover the potential, if this assumption was not perfectly satisfied, i.e. the \MAPs true DF does not perfectly follow a qDF. We considered two cases: a) a hot DF, that has less stars at small radii and more stars with low velocities than predicted by the qDF (reddish data sets in Fig. \ref{fig:isoSphFlexMix_mockdata}), or b) a cool DF that has broader velocity dispersion wings and less stars at large radii than predicted by the qDF (bluish data sets). We find that case a) would give more reliable results for the potential parameter recovery.
\\If we assumed that the distribution of stars from one \MAP is caused by radial migration away from the initial location of star formation, it would more likely that the qDF overestimates the true number of stars at smaller radii than underestimating it at larger radii. [TO DO: Is this actually a sensible argument???]
\\Following this, focusing the analysis especially on hotter \MAPs could be an advisable way to go in the future, if there is doubt that the stars truly follow the qDF.

\paragraph{Pollution of \MAPs due to binning.} Another critical point is the binning of stars into \MAPs depending on their metallicity and $\alpha$ abundances. Example 2 in \S ??? could be understood as a model scenario for decreasing bin sizes in the metallicity-$\alpha$ plane when sorting stars in different \MAPs, assuming that there is a smooth variation of qDF within the metallicity-$\alpha$ plane and each \MAP indeed follows a qDF. We find that, in the case of 20,000 stars in each bin, differences of $20\%$ in the qDF parameters of two neighbouring bins can still give quite good constraints on the potential parameters. 
\\We compare this with the relative difference in the qDF parameters in the bins in Fig. 6 of \cite{bov13}, which have sizes of $[Fe/H] = 0.1$ dex and $\Delta [\alpha/Fe] = 0.05$ dex. It seems that these bin sizes are large enough to make sure that $\sigma_{R,0}$ and $\sigma_{z,0}$ of neighbouring \MAPs do not differ more than $20\%$. Fig. \ref{fig:isoSphFlexMixCont} and \ref{fig:isoSphFlexMixDiff} suggest that especially the tracer scale length $h_R$ needs to be recovered to get the potential right. For this parameter however the bin sizes in fig. 6 of \cite{bov13} might not yet be small enough to ensure no more than $20\%$ of difference in neighbouring $h_R$. This is especially the case in the low-$\alpha$ ($[\alpha/Fe] \lesssim 0.2$), intermediate-metallicity ($[Fe/H] \sim -0.5$) \MAPs. The above is valid for 20,000 stars per \MAP. In case there are less than 20,000 stars in each bin the constraints are less tight and due to Poisson noise one could also allow larger differences in neighbouring \MAPs while still getting reliable results. [TO DO: Discuss Binney \& Sanders doubts about binning.]

\paragraph{Misjudgment of the parametric form of the potential.} In the long run \RM should incorporate a family of gravitational potential models that can reproduce the essential features of the MW's true mass distribution. While our fundamental assumption of the Galaxy's axisymmetry is at odds with the obvious existence of non-axisymmetries in the MW, we will not dive into investigating this implications in the scope of this paper. Instead we test how a misjudgment of the parametric potential form affects the recovery by fitting a potential of St\"{a}ckel form \citep{bat94} to mock data from a different potential family with halo, bulge and exponential disk. The recovery is quite successful and we get the best fit within the limits of the model. However, even a strongly flattened St\"{a}ckel potential component has difficulties to recover the very flattened mass distribution of an exponential disk. This will lead to underestimation of the vertical velocity dispersion at the sun. aA the qDF parameter $\sigma_{z,0}$ corresponds to the physical vertical velocity dispersion at the sun, a comparison with direct measurements could be a valuable cross-checking reference. [TO DO: Check, if this is actually true. Somehow my data sets do not have the expected vertical velocity dispersion at the sun...] In case of as many as 20,000 stars we should therefore already be able to distinguish between different potential models.
\\The advantage of using a St\"{a}ckel potential with \RM is firstly the exact and fast action calculation via the numerical evaluation of a single integral, and secondly that the potential has a simple analytic form, which greatly speeds up calculations of forces and frequencies (as compared to potentials in which only the density has an easy description like the exponential disk). A superposition of several simple Kuzmin-Kutuzov St\"{a}ckel components can successfully produce MW-like rotation curves (see \citet{bat94}, \citet{fam03} and Fig. ???) and one could think of adding even more components for more flexibility, e.g. a small roundish component for the bulge.
\\The potential model used by \citet{bov13} had only two free parameters (disk scale lentgh and halo contribution to $v_\text{circ}(R_\odot)$. To circumvent the obvious disadvantage of this being at all not flexible enough, they fitted the potential separately for each \MAP and recovered the mass distribution for each \MAP only at that radius for which it was best constrained - assuming that \MAPs of different scale length would probe different regions of the Galaxy best. Based on our results in Fig. \ref{fig:MW14vsKKS2SphFlex} this seems to be indeed a sensible approach [TO DO: Check that this is indeed the case - it is not clear to me from the plot. ???].
\\We suggest that combining the flexibility and computational advantages of a superposition of several St\"{a}ckel potential components with probing the potential in different regions with different \MAPs as done by \citet{bov13}, could be a promising approach to get the best possible constraints on the MW's potential.

\begin{itemize}
\item \textbf{Other Action/DF-based modelling} Compare to our own approach

\begin{itemize}
\item \underline{Piffl et al. 2014:} They fitted a superposition of DFs (superposition of qDFs for cohorts in thin disk, single qDF for thick disk, other DF for halo) to the full RAVE data set. No chemical information at all. No treatment of selection function. Fitting of velocity dispersion histograms in density bins. Advantage: Avoiding problem of accurate normalisation calculation. Disadvantage: A lot of stellar information is not used.
\item  \underline{Sanders \& Binney 2015:} developed extended distribution functions, i.e. functions of both actions and metallicity for thin/thick disk + halo. Including of selection function. BUT: potential not fitted.
\item \underline{Ultimate goal:} Fitting both a very flexible distribution function in both action/abundance space for the whole galaxy + flexible potential.
\item \underline{Binney's group's focus:} rather on developing eDFs, potential recovery more secondary.
\item \underline{Our focus:} getting very good constraints on potential with an optimum of simplicity in DF and flexibility in capturing the actual distribution. 
\item \underline{Our philosophy:} We see our approach as intermediate/first step before ultimately using eDFs. We don't think that there is an easy to find and formulate eDF that describes the distribution of stars in both action and abundance space (metallicity AND alpha), see fig. 6 in Bovy \& Rix (2013). Bovy \& Rix 2013 were fitting each \MAP separately with a qDF and potential and used the best fit potential model only to give a mass constraint at one radius. By doing so they/we 
\begin{itemize}
\item[a)] separate out complexity / avoid explicitely dealing with the substructure of the DF in abundance space while still including abundance information (it's easier to see what goes right and wrong; less assumptions) 
\item[b)] make use of different \MAPs constraining different regions of the potential best and are therefore less limited by having to get the potential model right.
\item[c)] Quote HW: "It provides true cross-checking redundancy w.r.t. the potential estimates." [What does he mean by that?]
\end{itemize}
\end{itemize}

\item \textbf{On the assumption of axisymmetry} 
\begin{itemize}
\item \underline{Key assumption of the modelling:} axisymmetry of the MW. Needed for action calculation/conservation.
\item \underline{Reality:}  real disk has spiral arms and ring-like structures, with a warp and a flare in the outer disk. Also the Milky Way's halo has substructure, streams and shell-like overdensities. In the case of non-axisymmetry actions are not conserved anymore, so this could indeed affect the modelling.
\item \underline{Does modelling still work?} Has to be investigated in detail in future work, by trying to recover the potential in N-body simulations. Because actions are at least conserved under adiabatic chnages of the potential, and the vertical action under radial migration, there is some hope, that it could still work.
\item \underline{Intermediate step to Ultimate goal:} Ultimate goal: Finding substructures observationally and describing theoretically the structure and evolution of potential perturbations. Including non-axisymmetries also in modelling. Could be approached as applying perturbations to an equilibrium model - the axisymmetric model. 
\item \underline{Other applications:} a) Axisymmetric model also helps finding and explaining sub-structures.  b) As we are still far away from a completely realistic MW potential model, the axisymmetric case will be our reference to turn x,v into orbits, needed as tracers of Galaxy evolution.
\end{itemize}

\item \textbf{What one could do with such an axisymmetric smooth model} $\longrightarrow$ finding substructure in the Disk


\end{itemize}
